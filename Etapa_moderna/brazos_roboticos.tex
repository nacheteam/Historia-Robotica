\subsubsection{Stanford Arm}

Los brazos robóticos han sido un gran pilar en el desarrollo de la robótica por su gran utilidad. Actualmente rara es la fábrica que no emplea brazos robóticos para hacer la producción mucho más ágil e incluso segura para los operadores de la misma. Este camino fue iniciado en el año 1969 en la Universidad de Stanford por Victor Scheinman.

\vspace{10px}

El primer brazo robótico fue desarrollado bajo el proyecto Hand-Eye en los laboratorios de Inteligencia Artificial de la Universidad de Stanford. El Stanford Arm fue diseñado tras los intentos de hacer brazos operables mediante un humano desempeñados en dicha Universidad. Los intentos precedentes incluían modificaciones de brazos ortopédicos e incluso un modelo hidráulico que resultaba muy peligroso por la rapidez de sus movimientos. El Stanford Arm se creó sin tener una forma antropomórfica y con 6 grados de libertad manipulado de forma completamente eléctrico.

\vspace{10px}

El robot estaba controlado mediante cámaras y elementos parecidos a joysticks gracias a que el brazo llevaba incorporados potenciómetros y sensores que permitían el control del mismo.
