\subsubsection{SRI}
SRI o Stanford Research Institute fue una institución amparada bajo la Universidad de Stanford de propósito general en un inicio. Su fundación fue dificultosa y no fue hasta el tercer intento cuando se consiguió en el año 1946.

\vspace{10px}

El SRI hizo investigaciones en variados campos como agricultura, ejército o contaminación pero no fue hasta 1966 cuando el Centro de Inteligencia Artificial comenzó a funcionar bajo el mando de Charles Rosen. El primer proyecto realizado por el SRI en inteligencia artificial fue el Robot Shakey, el primer robot cuya intención no era sólo moverse, si no razonar. Este robot era capaz de analizar el entorno mediante una cámara y procesaba el lenguaje natural para recibir instrucciones. Desarrollando las habilidades de este robot en el desplazamiento por una habitación esquivando los obstáculos que se le imponían en la misma se desarrollaron algoritmos de Visión por Computador tales como filtros y convoluciones para detección de bordes de objetos a ser esquivados y además se desarrollaron algoritmos tan importantes como el A*, el algoritmo por excelencia de cálculo de posibles caminos.

\vspace{10px}

Shakey fue dado por finalizado en el año 1972, siendo ahora los esfuerzos del SRI dirigidos en la creación del Internet para suceder a ARPANET. La primera conexión que se produjo entre dos ordenadores usando el Internet como lo conocemos ahora se produjo en 1977 entre los laboratorios del SRI y la universidad de California UCLA.

\vspace{10px}

Tras estos grandes hitos el SRI se desprendió de la Universidad de Stanford para comenzar su andadura como una empresa, fundándose en ese momento la compañía SRI International, pero la Inteligencia Artificial ya había comenzado.

\subsubsection{Laboratorios de IA del MIT}
En la misma época el Instituto Tencnológico de Massachusetts o MIT abrió sus laboratorios de inteligencia artificial donde surgieron algunas de las figuras más reconocidas en el ámbito de la computación.

\vspace{10px}

El inicio de la investigación en el ámbito de la computación en el MIT comenzó con figuras tan reseñables como Shannon o Vannevar Bush, el ideólogo del analizador diferencial. En sus comienzos el laboratorio no empezó formándose como tal, si no que se creó un proyecto llamado MAC (Man and Computer) en el que estaban figuras tan importantes como John McCarthy el inventor de Lisp. En un principio el proyecto se dirigió intentando abarcar un amplio rango de temas, como son: la visión por computador, el movimiento mecánico y la manipulación de objetos a partir de máquinas y el manejo del lenguaje natural.

\vspace{10px}
De los despachos en los cuales se llevaban a cabo estas investigaciones surgieron otras figuras como Richard Stallman (creador del proyecto GNU) e invenciones como las máquinas Lisp (máquinas que ejecutaban muy eficientemente Lisp).
