\subsubsection{Introducción}

Tras el desarrollo de robots primigenios con funcionalidad concreta pero fija se planteó la posibilidad de integrar movimiento por el terreno en los mismos para poder realizar nuevas tareas u otras de forma más eficiente. En este ámbito se invirtió para lograr avances en esquivar objetos del escenario o en sistemas para andar basados en animales cuadrúpedos o bípedos como nosotros. A continuación analizamos de forma cronológica estos hechos.

\subsubsection{RB5X de RB Robot Corporation}
Un inicio en la movilidad en los robots fue incorporarles ruedas u orugas como herramientas para su desplazamiento por el suelo. Estos tipos de movimientos eran básicos, tales como hacia delante, hacia atrás o hacia los lados. Un avance en el movimiento inteligente lo protagonizó el Stanford's Cart, el cual fue progamado con sensores para desplazarse de una punta a otra de un recinto esquivando los objetos a su paso. Este movimiento inteligente era realizado gracias a un sistema de Visión por Computador que incluía el mismo. Tras esto surgieron nuevos intentos como el RB5X, un robot de pequeñas dimensiones capaz de desplazarse y evitar obstáculos o, en caso de no poder evitarlos, capaz de reconstruir su trayectoria tras impactar con algún objeto.

El robot se ideó de forma multipropósito, puesto que los módulos que integraba eran programables por lo que las empresas tomaron el diseño y lo adecuaron a las tareas que se querían. El robot tuvo variantes tales como asistentes personales a los cuales les podías dar órdenes como recoger el periódico y a través de un brazo robótico que incluía poder interactuar con el medio. Además se le desarrolló una interfaz oral básica para poder darles las órdenes mediante la voz. Además era capaz de aprender de su entorno.
