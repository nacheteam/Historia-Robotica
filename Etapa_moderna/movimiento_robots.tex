\subsubsection{Introducción}

Tras el desarrollo de robots primigenios con funcionalidad concreta pero fija se planteó la posibilidad de integrar movimiento por el terreno en los mismos para poder realizar nuevas tareas u otras de forma más eficiente. En este ámbito se invirtió para lograr avances en esquivar objetos del escenario o en sistemas para andar basados en animales cuadrúpedos o bípedos como nosotros. A continuación analizamos de forma cronológica estos hechos.

\subsubsection{RB5X de RB Robot Corporation}
Un inicio en la movilidad en los robots fue incorporarles ruedas u orugas como herramientas para su desplazamiento por el suelo. Estos tipos de movimientos eran básicos, tales como hacia delante, hacia atrás o hacia los lados. Un avance en el movimiento inteligente lo protagonizó el Stanford's Cart, el cual fue progamado con sensores para desplazarse de una punta a otra de un recinto esquivando los objetos a su paso. Este movimiento inteligente era realizado gracias a un sistema de Visión por Computador que incluía el mismo. Tras esto surgieron nuevos intentos como el RB5X, un robot de pequeñas dimensiones capaz de desplazarse y evitar obstáculos o, en caso de no poder evitarlos, capaz de reconstruir su trayectoria tras impactar con algún objeto.

\vspace{10px}

El robot se ideó de forma multipropósito, puesto que los módulos que integraba eran programables por lo que las empresas tomaron el diseño y lo adecuaron a las tareas que se querían. El robot tuvo variantes tales como asistentes personales a los cuales les podías dar órdenes como recoger el periódico y a través de un brazo robótico que incluía poder interactuar con el medio. Además se le desarrolló una interfaz oral básica para poder darles las órdenes mediante la voz. Además era capaz de aprender de su entorno.

\subsubsection{Phony Pony}

Tras los intentos de hacer robots que incorporasen ruedas u orugas la Universidad de California del Sur realizó el desarrollo del que se conoce como primer robot cuadrúpedo llamado Phony Pony. El diseño de este robot se realizó en 1968 (anterior a RB5X) por los profesores e investigadores Frank y McGhee.

\vspace{10px}

El robot intentaba imitar las articulaciones de los animáles cuadrúpedos que conocemos de forma que lo desarrollaron con dos articulaciones: la cadera y la rodilla. De esta forma el robot podía mover la pierna hacia delante y además flexionarla. La velocidad del movimiento de las patas era extremadamente lenta, ya que al igual que ocurría con el Stanford Arm (contemporáneo a Phony Pony) los elementos eléctricos empleados en el movimiento estaban muy limitados. El Phony Pony era capaz de imitar comportamientos tales como andar, arrastrarse, agacharse o trotar.

\vspace{10px}

El control de este robot se realizaba de forma remota gracias a la tecnología desarrollada ya en esa época. Internamente el robot estaba diseñado mediante patrones de movimiento, de forma que estaba implementado con un autómata finito que tomaba las transiciones del mismo tras recibir la entrada del mando de control.

\subsubsection{WAP1}
En primer paso que se dió entorno a la idea de realizar un robot bípedo tuvo lugar en 1969 en la Universidad de Waseda en Japón. El robot WAP1 (Waseda's antropomorphic pneumatically-activated pedipulators) fue diseñado por el doctor Ichiro Kato.

\vspace{10px}

El doctor Ichiro Kato trabajaba por aquel momento en el terreno médico y empezó a intentar desarrollar músculos artificiales para poder sustituir músculos defectuosos en sus pacientes. El estudio desarrrolló una serie de dispositivos que al hincharse adquirían las formas que ellos necesitaban para simular el comportamiento de un músculo. Estas investigaciones darían sus frutos plasmándose en el primer robot bípedo creado. Los músculos que permitían el movimiento de WAP1 estaban hechos de goma y se inflaban con actuadores neumáticos. Este primer robot no era capaz más que de andar muy lentamente controlando el flujo de aire que se llevaba hacia las piezas de goma. Estas técnicas hacían que WAP llegara a tardar mucho tiempo en completar un paso, además de que no estaba pensado para andar una distancia dando pasos de forma consecutiva.

\vspace{10px}

En los dos siguientes años el doctor Ichiro Kato mejoró el diseño del WAP1 creado el WAP2 y el WAP3. La segunda versión de estos robots tuvo una mejora en la fuerza que los músculos eran capaces de aplicar sobre la estructura del mismo, pudiendo ahora levantar más peso y tener una estructura más robusta. La última versión fue desarrollada en 1971 y en esta el robot ya era capaz de rotar su movimiento, por lo que ya no tenían que ser movimientos rectos y, además, era capaz de superar pequeños obstáculos tales como ligeras inclinaciones o incluso escaleras de pequeño tamaño. Para poder desarrollar todo esto no sólo fueron necesarios los músculos artificiales del doctor Kato, si no además sus estudios en el control y mejora de la postura con lo que fueron capaces de controlar el centro de gravedad del robot, cosa completamente imprescindible para caminar sobre dos piernas.

\subsubsection{WL-9DR}
Tras los avances del doctor Ichiro Kato se intentó agilizar el movimiento de estos robots, puesto que el tiempo que había que esperar hasta que el WAP era capaz de avanzar un paso hacía que no fuera práctico su uso, ya que los robots que empleaban ruedas por ejemplo eran mucho más ágiles. Entre los años 1979 y 1980 la Universidad de Waseda desarrolló el WL-9DR bajo la dirección del doctor Kato, al igual que con sus predecesores los WAP.

\vspace{10px}

El funcionamiento de esta máquina también era mediante actuadores hidráulicos que operaban más rápidamente que los diseñados para WAP. Así mismo el control de este robot se hacía mediante un procesador de 16 bit conectado al robot mediante cables. El WL-9RD era capaz de dar un paso en 10 segundos, además de realizar ciclos completos andando, es decir, encadenar pasos para avanzar desde el punto de partida a un punto meta que se le fijaba. La velocidad con la que avanzaba era tan lenta que sus creadores no podían podían denominar al robot como dinámico, por lo que se le conoce como el primer robot bípedo cuasi-dinámico.

\vspace{10px}

El robot tenía un sistema de pesos que mantenía el equilibrio del mismo como se requiriera, por ejemplo si el robot estaba estático se centraba el punto de apoyo y se mantenía la postura tanto por la gran base rectangular de sus pies como por los pesos centrados. Si se quería dar un paso hacia delante el robot alzaba alguna de las piernas y se cambiaba el centro de los pesos un poco hacia delante, de forma que el robot se inclinara ligeramente para seguir con el siguiente paso. Para poder andar solamente 0.5 metros el robot requería sobre una docena de pasos llevándole entorno a un minuto completar dicha distancia.

\subsubsection{Aquarobot}
Tras los robots anteriormente mencionados (que sentaron precedentes en el tema de caminar) se siguieron desarrollando las técnicas que se empleaban en el equilibrio y las partes móviles de los mismos tal y como veremos en las siguientes secciones. Entre estos robots surgieron algunos proyectos que intentaron ir un poco más allá como el Aquarobot cuyo objetivo era ser capaz de andar bajo el agua.

En el instituto de investigaciones de puertos de Japón se pensó en que sería útil que los operarios que tenían que trabajar bajo el agua dispusieran de facilidades para hacer la tarea más rápida y sencilla. Una de los trabajos que más tiempo llevaban en el puerto eran los de inspección bajo el agua para certificar el estado de las estructuras o diagnosticar algún tipo de fallo que se debía solventar. Por ello en el año 1984 se desarrolló la primera versión del Aquarobot, llegando a tener este hasta tres versiones diferentes haciendo mejoras sobre lo ya propuesto.

El diseño básico del robot era una estructura con 6 patas capaz de andar bajo el agua. Los motores que empleaban eran de corriente continua (no neumáticos) colocados en cada una de las patas que iban selladas de forma estanca para impedir la entrada del agua. Además el robot era controlado por un pequeño procesador incluido en el mismo al que se le conectaban cables para comunicarlo con una computadora o un mando de control. El Aquarobot disponía asimismo de una cámara integrada que permitía ver en tiempo real lo que el robot observaba, de forma que los técnicos podían visualizar el estado del puerto sin necesidad de bajar ellos al agua.

Cada pata tenía 3 grados de libertad con los que el robot podía andar y rotar sobre sí mismo. El peso aproximado de este robot era de unos 857 kilogramos en su primera versión y de unos 280 en la tercera. El sistema con el que se manipulaba al robot y los patrones de movimiento que se le definían en su propio procesador estaba programados en C++. El tiempo en el que el robot era capaz de responder a las órdenes que se le daban era de unos 50 milisegundos, gracias a la conexión que se tenía mediante cable.
