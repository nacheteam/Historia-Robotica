\documentclass[a4paper,11pt]{article}

\usepackage[spanish]{babel}
\usepackage[utf8]{inputenc}
\usepackage{enumerate}
\usepackage{amsmath}
\usepackage{extsizes}
\usepackage{amssymb}
\usepackage{dsfont}
\usepackage{graphicx}
\usepackage{cancel}
\usepackage[usenames]{color}
\usepackage[dvipsnames]{xcolor}
\usepackage{accents}
\usepackage{flushend}
\usepackage{tikz}
\usepackage[LGR,T1]{fontenc}
\newcommand{\textgreek}[1]{\begingroup\fontencoding{LGR}\selectfont#1\endgroup}
\usetikzlibrary{arrows,automata}
\usepackage{multicol}
\setlength{\columnsep}{1cm}
\usepackage{listings}



\usepackage[hidelinks]{hyperref}

\usepackage[vmargin=3cm,hmargin=3cm]{geometry}
%\setlength\parindent{0pt}

\setlength\parindent{0pt}


% Carpeta con las imágenes
%\graphicspath{{}}

\begin{document}


	\begin{center}
		\large{\textbf{Historia de las Matemáticas (2018-2019)} \\ Grado en Ingeniería Informática y Matemáticas \\ Universidad de Granada }
		\vspace*{2.5cm}

		\rule{\textwidth}{1.6pt}\vspace*{-\baselineskip}\vspace*{4pt}
		\rule{\textwidth}{1.6pt}\vspace*{-\baselineskip}\vspace*{2pt}
		\vspace{0.5cm}

		\Huge{Historia de la Robótica}

		\vspace{0.5cm}
		\rule{\textwidth}{1.6pt}\vspace*{-\baselineskip}\vspace*{2pt}
		\rule{\textwidth}{1.6pt}\vspace*{-\baselineskip}\vspace*{4pt}

		\vspace{2cm}

\begin{figure}[h!]
	\centering
	\includegraphics[scale=0.5]{TeoriaDeAutomatas/img/logoUgrCiencias}
	\label{fig:logougrciencias}
\end{figure}

		\vspace{1.5cm}
		\large{Ignacio Aguilera Martos, Alicia Rodríguez Gómez, Darío Sierra Martínez \\ \today }

	\end{center}


	\newpage
\begin{abstract}
	La computación ha tenido un desarrollo muy acelerado desde el siglo XX a nuestros días tomando para ello diferentes objetivos. En primer lugar se intentó automatizar tareas mediante complicadas máquinas que comenzaron siendo completamente mecánicas para proseguir con las actuales electrónicas. En el transcurso de la historia este intento de automatizar las tareas repetitivas y tediosas ha venido acompañado del desarrollo de tecnologías y campos aledaños como la teoría de autómatas o la inteligencia artificial para lograr lo que hoy en día conocemos como robots, o lo que es lo mismo, sistemas artificiales diseñados con un propósito propio. En este trabajo nos proponemos el estudio de la teoría y avances relacionados con la robótica acompañados de un análisis y revisión de la historia de los robots desde los primeros autómatas diseñados en la época griega hasta robots emocionales e inteligentes como Sophia.
\end{abstract}

\newpage

\tableofcontents

\newpage

\section{Teoría de Autómatas, Automatización e Inteligencia Artificial}



En esta parte abarcaremos qué es un robot, la posible equivalencia entre robot y autómata y la evolución de estos conceptos a lo largo del tiempo. Es decir claro queda que un robot debe ser un tipo de máquina, pero la noción de lo que es una máquina ha variado significativamente a lo largo del tiempo. \\

Nadie dudaría sobre la veracidad de la siguiente afirmación 'Un ordenador es una máquina'; sin embargo, hay una diferencia cualitativa importante entre un ordenador y un actuador mecánico, por ejemplo: una puerta automática. \\

La noción de automatismo será clave para discernir qué es o no es un robot. \\

\subsection{ Sobre la noción de máquina}


Partimos del concepto más simple hasta ahora mencionado, 'la máquina' y de entre ellas las más sencillas de entender, las máquinas aristotélicas, siglo III A.C: \\

-Simple machine, any of several devices with few or no moving parts that are used to modify motion and force in order to perform work. (Enciclopedia Britannica) \\


Serían: El plano inclinado, La palanca, La polea, El torno, La cuña y el tornillo. El primer enfoque de qué es una máquina está mucho más relacionado con el equilibro de fuerzas en un sistema estático que con la abstracta definición de qué cosas son máquinas que tenemos hoy en día. 

Llegamos al siglo XVI, la idea de una máquina es la de un objeto que cuyas partes distinguibles pueden ser clasificadas como una de las anteriores máquinas simples. \\

- 'Thus Leonardo was the first to advocate the necessity of a science of mechanisms.' -  \\

- ' A book about the nature of mechanism must precede a book about their aplications' \\

Ladislao Retti: 'The Unknown Leonardo' \\

Este pensamiento hace avanzar la 'ciencia de la máquina' , en los siglos anteriores el estudio de las máquinas era empírico alejado de cualquier conceptualización teórica. Es decir, el conjunto de las máquinas estaba definido por extensión. Este paso cualitativo permitió el comienzo de la conceptualización teórica de las máquinas.

Vemos el auge de este pensamiento en el mecanicismo filosófico. Encontramos grandes matemáticos como Descartes o Newton participando en esta teoría. El concepto de máquina evoluciona, una máquina será aquel objeto descriptible mediante las leyes de la física, la mecánica (clásica) concretamente. Vemos una definición intensiva en contraposición al pensamiento anterior. Nace con Descartes el pensamiento de que una máquina no es necesariamente un objeto artificial, considerando los animales de manera mecánica, complicada, sí, pero máquina \textit{per se}. \\

La evolución del termino 'maquina' se va alejando cada vez más de la aplicación concreta de la 'maquina' en sí, hacia el concepto de que una máquina es aquel objeto cuyo comportamiento es determinista. \\

- A machine (or mechanical device) is a mechanical structure that uses power to apply forces and control movement to perform an intended action (Wikipedia) \\

A nuestros ojos, la definición parece incompleta por lo pronto, pues en el conjunto que define encontraríamos las máquinas simples,los coches pero no los ordenadores. ¿Es que acaso no son máquinas? Si bien es cierto que un ordenador puede comportarse de este modo, no es una cualidad intrínseca de un ordenador. Es decir: podemos conectar un actuador mecánico para que cumpla la definición, pero no tiene mucho sentido que al desconectárselo deje de ser una máquina. \\

El elemento invariante es su cualidad determinística, una máquina actúa conforme a un patrón. La categorización de un objeto físico como una máquina se produce cuando de su comportamiento se elimina el azar, se puede conocer su estado exacto después de recibir un estímulo.



\subsubsection{ ¿Qué es un autómata? (mecánica)}

Queda claro que un autómata debería ser algún tipo de máquina, es decir: en esencia debería presentar un comportamiento determinista. Aquí cobra importancia la noción de 'automatización'.

Autómata proviene de la palabra griega \textbf{\textgreek{αὐτόματος}} , cuya traducción seria '\textbf{que se mueve por si mismo}'. Es decir, la forma clásica de conceptualizar un autómata sería justo la de automatización, sería por tanto una máquina capaz de actuar sin intervención humana. Dejaríamos fuera de este conjunto a todas la máquinas simples y a los coches por ejemplo, pues no son capaces de llevar a cabo su función sin una intervencion expresa.




\subsection{ Conceptualización de la máquina y el autómata.}

Hemos hablado de lo complicado que resulta dar una definición completa de estos conceptos, pues aunque todos sepamos qué es una maquina por extensión, las cualidades que comparten los objetos que llamamos máquinas pueden llegar a ser muy excasas.

\subsubsection{ Algebras de Boole y sistemas combinacionales}

En el siglo XIX se da otro gran avance cualitativo sobre el entendimiento de qué cosas son máquinas, alejándonos de la definición "mecánica" de los mecanicistas y entrando en el campo de la lógica.

Las algebras de Boole son la base teórica de los sistemas combinacionales. Un sistema combinacional es una máquina (según hemos definido) cuya salida es únicamente función de su entrada, es decir, no presenta memoria sobre ningún estado pasado. Este será un aspecto fundamental en el avance de las "máquinas", su memoria, la cualidad de tener memoria sobre un uso anterior marcará toda una escala de complejidad.


La realización de este concepto se ve con gran claridad en los circuitos lógicos, a partir de operaciones simples ( como son la negación, la disyunción y la conjunción ) podemos llegar a expresar teóricamente cuál sería el funcionamiento de un sistema lo suficientemente complejo como para realizar operaciones artiméticas a mayor velocidad que un humano.

Queremos destacar que aunque estos sistemas se consideran típicamente como digitales, hechos a base de transistores recibiendo impulsos eléctricos, no tienen una diferencia cualitativa importante en cuanto a capacidad de computo o expresividad con un sistema mecánico clásico.

<!-- TODO  ejemplo de sist combinacional -->

%En 1642 el filósfo y matemático francés Blaise Pascal inventó una calculadora mecánica, *La pascalina*. Aunque es evidente que hay una gran diferencia entre una calculadora mecánica y una digital, ambas pueden conceptualizarse como sistemas combinacionales; ya que ninguna de las dos tiene, memoria sobre ejecuciones anteriores (estados) y el sistema, sean cuales sean sus actuadores, produce una salida como función única de una entrada. -->


\subsection{ Autómatas finitos}


En el siglo XX nace la teoría de autómatas como siguiente nivel de conceptualización sobre qué son las máquinas y qué pueden hacer, la pregunta fundamental de este momento histórico es : ¿Qué problemas puede resolver una máquina?.

Daremos ahora una definición de autómata alejada del mecanicismo anterior, más parecido al modelo abstracto de las algebras de Boole: Un autómata finito es una máquina (abstracta) cuya salida no depende solo de la entrada actual.

Un autómata finito tiene estados, posiciones de memoria que dependerán de las entradas anteriores, esto eleva la capacidad expresiva del autómata muy por encima de la del S.C. \\

- En electrónica un autómata es un sistema secuencial, aunque en ocasiones la palabra es utilizada también para referirse a un robot. [...] Sin embargo, la rápida evolución de los autómatas hace que esta definición no esté cerrada. (Wikipedia)- \\

Aquí un sistema secuencial es justo lo que hemos definido como autómata, un sistema combinacional dotado de memoria.  Un ejemplo muy sencillo de un cálculo que podría hacerse con un autómata finito y no bastaría con un sistema combinacional sería la tarea de decidir si un conjunto de ceros es par o impar. Esto puede parecer una tarea trivial pero es un cálculo recurrente en la comprobación de error.

<!-- Meter autómata -->

<!-- TODO: Ejemplo de primeros  autómatas -->

\subsubsection{Aumentando la abstracción: Lenguajes y problemas }


Esta evolución sobre lo que era una máquina y qué problemas podía resolver planteó el mismo porblema que habíamos subrayado antes. ¿Si podemos clasificar las máquinas que tenemos según su complejidad, podríamos hacerlo con los problemas?

Este es fundamentalmente el objeto de estudio de la teoría de autómatas, pues responde fundamentalmente a la pregunta: ¿ Qué máquina necesito para resolver este problema?

La definición del lenguaje formal como se hace en esta teoría es bastante natural:


Sea  $\mathcal{A}$ un conjunto finito de símbolos, a este conjunto lo llamaremos alfabeto, muy en consonancia con el uso popular del concepto. A los símbolos de $\mathcal{A}$ se les llamará letras, a las secuencias finitas de letras, palabras, y a todas las palabras que se puedan formar con las letras se las conoce como lenguaje generado por $\mathcal{A}$ o por sintetizar $\mathcal{A}^{*}$. Un lenguaje sobre $\mathcal{A}$, será un subconjunto de su lenguaje generado. Esto es:


$$ \mathcal{L} \text{ es un lenguaje sobre } \mathcal{A}  \iff \mathcal{L} \subset \mathcal{A^{*}} $$


Hasta aquí existe un paralelismo evidente con lo que asociaríamos con un lenguaje de manera popular. Pues bien, existe una relación entre las categorías de complejidad que mencionamos antes y los lenguajes. \\ \newpage

Por ejemplo, definamos el siguiente lenguaje, palabras sobre $\{0,1\}$ tales que tengan un número par de 0's.

\begin{equation}
\mathcal{L} = \{\omega \in \{0,1\}^{*} : \text{ $\omega$ tiene un nº par de 0's} \}
\end{equation}


\begin{center}
	\begin{tikzpicture}[scale=0.2]
	\tikzstyle{every node}+=[inner sep=0pt]
	\draw [black] (40.6,-19.3) circle (3);
	\draw (40.6,-19.3) node {$q_1$};
	\draw [black] (40.6,-19.3) circle (2.4);
	\draw [black] (20.7,-19.3) circle (3);
	\draw (20.7,-19.3) node {$q_0$};
	\draw [black] (37.6,-19.3) -- (23.7,-19.3);
	\fill [black] (23.7,-19.3) -- (24.5,-19.8) -- (24.5,-18.8);
	\draw [black] (23.7,-19.3) -- (37.6,-19.3);
	\fill [black] (37.6,-19.3) -- (36.8,-18.8) -- (36.8,-19.8);
	\draw (30.65,-19.8) node [below] {$0$};
	\draw [black] (50.8,-19.3) -- (43.6,-19.3);
	\draw (51.3,-19.3) node [right] {$\epsilon$};
	\fill [black] (43.6,-19.3) -- (44.4,-19.8) -- (44.4,-18.8);
	\end{tikzpicture}
\end{center}

%TODO citar el libro que has sacado de la biblioteca

El siguiente autómata sería capaz de clasificar las palabras sobre el lenguaje generado por $\{0,1\}^{*}$ de manera que podría decirnos qué palabras pertenecen exactamente al lenguaje. La correspondencia entre lenguaje y autómata es aun mayor, porque de hecho, este autómata solo acepta las palabras del lenguaje $\mathcal{L}$. Una vez introducidos los conceptos daremos una definición rigurosa. \\

Un autómata se modela de la siguiente manera, es una tupla de 5 elementos: $(Q, \mathcal{A}, q_0, \delta, F)$

\begin{itemize}
	\item $Q$ : Conjunto finito de estados que numeramos $ q_0 \dots q_n$
	\item $\mathcal{A}$: Alfabeto, conjunto finito de símbolos sobre el que se crearán palabras del lenguaje.
	\item $q_0$ el estado inicial.
	\item $\delta :Q\times \mathcal{A} \rightarrow Q$. Función que nos indica como leer la palabra.
	\item $F$ : Conjunto de estados finales
\end{itemize}


Explicaremos ahora con detenimiento la relación entre los lenguajes y los autómatas pues son dos maneras de hablar del mismo concepto. Supongamos el siguiente lenguaje sobre $\{0,1\}$: Si la palabra empieza por 1 no pertenece al lenguaje, si empieza por 0 sí. El autómata que únicamente acepta las palabras de este lenguaje es el siguiente:

\begin{center}
	\begin{tikzpicture}[scale=0.2]
	\tikzstyle{every node}+=[inner sep=0pt]
	\draw [black] (20.6,-27.7) circle (3);
	\draw (20.6,-27.7) node {$q_0$};
	\draw [black] (49.8,-27.7) circle (3);
	\draw (49.8,-27.7) node {$q_1$};
	\draw [black] (49.8,-27.7) circle (2.4);
	\draw [black] (38.7,-35.6) circle (3);
	\draw (38.7,-35.6) node {$q_2$};
	\draw [black] (23.527,-27.044) arc (101.19936:78.80064:60.1);
	\fill [black] (46.87,-27.04) -- (46.19,-26.4) -- (45.99,-27.38);
	\draw (35.2,-25.4) node [above] {$0$};
	\draw [black] (52.48,-26.377) arc (144:-144:2.25);
	\draw (57.05,-27.7) node [right] {$0,1$};
	\fill [black] (52.48,-29.02) -- (52.83,-29.9) -- (53.42,-29.09);
	\draw [black] (35.712,-35.827) arc (-90.34893:-136.81015:18.322);
	\fill [black] (35.71,-35.83) -- (34.92,-35.32) -- (34.91,-36.32);
	\draw (27.52,-34.81) node [below] {$1$};
	\draw [black] (41.148,-33.885) arc (152.74451:-135.25549:2.25);
	\draw (46.07,-34.38) node [right] {$0,1$};
	\fill [black] (41.55,-36.5) -- (42.03,-37.31) -- (42.49,-36.42);
	\draw [black] (9.6,-27.7) -- (17.6,-27.7);
	\draw (9.1,-27.7) node [left] {$\epsilon$};
	\fill [black] (17.6,-27.7) -- (16.8,-27.2) -- (16.8,-28.2);
	\end{tikzpicture}
\end{center}



Un autómata puede presentarse también a modo de grafo como vemos, los estados se representan como nodos y los arcos representan las transiciones que modela al función $\delta(\cdot)$.Es decir, leemos la palabra de izquierda a derecha y por cada símbolo de la palabra, buscamos cuál es la transición asociada al estado en que nos encontramos y el símbolo que estamos leyendo.

Ahora bien, uno de los estados está marcado con dos circunferencias, este es un estado final, al ser el único: $F = \{q_1\}$. El resto de los elementos de la tupla serían:

\begin{multicols}{2}
	\begin{itemize}
		\item $Q =\{q_0,q_1,q_2\}$
		\item $\mathcal{A} = \{0,1\}$
		\item $F = \{q_1\}$
	\end{itemize}
	
\end{multicols}

En cuanto a la función de transición, tendremos que el par $(a,q_i)$ pertenece al dominio de la función, si en el grafo existe un arco que salga de $q_i$ con etiqueta $a$. Las transiciones serían las siguientes:

\begin{multicols}{2}
	\begin{itemize}
		\item $\delta(0,q_0)=q_1$
		\item $\delta(1,q_0)=q_2$
		\item $\delta(a,q_1)=q_1 \ \forall a \in \{0,1\}$
		\item $\delta(a,q_2)=q_2 \ \forall a \in \{0,1\}$
	\end{itemize}
\end{multicols}

El símbolo $\epsilon$ se reserva para la palabra vacía, que es una entelequia que simboliza la palabra que no contiene ninguna letra.





\subsubsection{No-Determinismo}

Los autómatas que hemos examinado antes son deterministas, es decir, dada una entrada y partiendo de un estado, es trivial averiguar en qué estado acabaremos una vez leída. Sin embargo podemos considerar qué pasaría si nuestra función define las siguientes transiciones:

\begin{multicols}{2}
	\begin{itemize}
		\item $\delta(0,q_i)=q_j$
		\item $\delta(0,q_i)=q_k \hspace{2cm} j\neq k$
	\end{itemize}
\end{multicols}


Es decir: ¿Qué hacemos si estando el estado $q_i$ leemos un $0$? ¿Nos vamos a $q_j$ o a $q_k$? La respuesta no es simple. Los autómatas finitos no deterministas nacen por su enorme capacidad de expresión y su definición menos rigurosa que la de los anteriores. Sería una gran ventaja que fuesen equivalente, de manera que pudiésemos usar este último, más sencillo de definir, y aun así tener la potencia expresiva del anterior. La respuesta será que sí. \\


Su definición es parecida a la anterior salvo por un detalle, la función de transición:

$$\delta :Q\times \mathcal{A} \rightarrow \mathcal{P}(Q)$$

Aquí $\mathcal{P}(Q)$ es el conjunto potencia de $Q$, las partes de $Q$, es decir, dado un estado y un símbolo, podemos construir un autómata que tenga una arco conectando este estado $q_i$ con todos los demás incluyendo el mismo.


Supongamos el siguiente problema, queremos crear un autómata que nos distinga  si una palabra pertenece a este lenguaje:

$$\{ \omega \in \{0,1\}^* : \omega \text{ tiene un número par de 0's} \} \cup \{\omega  \in \{0,1\}^*: \omega \text{ contiene al cadena }  101 \}$$

\begin{center}
	\begin{tikzpicture}[scale=0.2]
	\tikzstyle{every node}+=[inner sep=0pt]
	\draw [black] (17.2,-29) circle (3);
	\draw (17.2,-29) node {$q_0$};
	\draw [black] (17.2,-29) circle (2.4);
	\draw [black] (28.7,-22.7) circle (3);
	\draw (28.7,-22.7) node {$q_1$};
	\draw [black] (28.7,-22.7) circle (2.4);
	\draw [black] (28.7,-33.3) circle (3);
	\draw (28.7,-33.3) node {$q_1'$};
	\draw [black] (41.7,-22) circle (3);
	\draw (41.7,-22) node {$q_2$};
	\draw [black] (42.7,-33.3) circle (3);
	\draw (42.7,-33.3) node {$q_2'$};
	\draw [black] (58.5,-33.3) circle (3);
	\draw (58.5,-33.3) node {$q_3'$};
	\draw [black] (58.5,-33.3) circle (2.4);
	\draw [black] (6.1,-29) -- (14.2,-29);
	\draw (5.6,-29) node [left] {$\epsilon$};
	\fill [black] (14.2,-29) -- (13.4,-28.5) -- (13.4,-29.5);
	\draw [black] (19.83,-27.56) -- (26.07,-24.14);
	\fill [black] (26.07,-24.14) -- (25.13,-24.09) -- (25.61,-24.96);
	\draw (23.95,-26.35) node [below] {$1$};
	\draw [black] (20.01,-30.05) -- (25.89,-32.25);
	\fill [black] (25.89,-32.25) -- (25.32,-31.5) -- (24.97,-32.44);
	\draw (22.01,-31.67) node [below] {$1$};
	\draw [black] (39.348,-23.841) arc (-61.15181:-112.68383:9.343);
	\fill [black] (39.35,-23.84) -- (38.41,-23.79) -- (38.89,-24.66);
	\draw (35.4,-25.54) node [below] {$0$};
	\draw [black] (31.7,-33.3) -- (39.7,-33.3);
	\fill [black] (39.7,-33.3) -- (38.9,-32.8) -- (38.9,-33.8);
	\draw (35.7,-33.8) node [below] {$0$};
	\draw [black] (45.7,-33.3) -- (55.5,-33.3);
	\fill [black] (55.5,-33.3) -- (54.7,-32.8) -- (54.7,-33.8);
	\draw (50.6,-33.8) node [below] {$1$};
	\draw [black] (30.625,-20.424) arc (128.5933:57.57106:7.686);
	\fill [black] (30.63,-20.42) -- (31.56,-20.32) -- (30.94,-19.53);
	\draw (34.95,-18.21) node [above] {$0$};
	\end{tikzpicture}
\end{center}

Si nos fijamos, los estados $q_i'$ y los $q_i'$ conforman 2 autómatas independientes, que hemos unido usando el no-determinismo en la transición $\delta(q_0,1)$ hemos dado un autómata que reconoce únicamente las palabras del lenguaje unión, uniendo los autómatas.

Esto se puede extender todavía más, introduciendo el concepto de las transiciones nulas. Una transición nula es la manera más cómoda de unir dos autómatas pues puedes moverte por libertad por todos los estados entre los que haya una transición nula. Intuitivamente esto quiere decir que podemos ir de un estado a otro sin leer el símbolo actual. 


\subsubsection{Ejemplos de Autómatas}

Después de estas consideraciones queda pensar,cuál es el papel de estos modelos en cuanto a su papel en temas de robótica. Pues pensemos en una función relativamente simple que querríamos que tuviese un robot, la comunicación por ejemplo.

Un protocolo de comunicación es de los requerimientos más fundamentales ya no en la robótica sino en cualquier sistema de relativa complejidad pues la función en sí es tan importante como la existencia de una interfaz de comunicación con el usuario. No nos referimos necesariamente a la capacidad de expresión del agente, sino a la existencia de un protocolo que le permita transmitir información de forma coherente y exacta con el humano con el que interaccione, aunque sea este su programador.

Hablamos por ejemplo, da la implementación del protocolo TCP/IP, que en el fondo, es una máquina de estados finita o como lo venimos llamando, un autómata finito determinista. Daremos un ejemplo de cómo modelar un sistema de autentificación de usuarios con una máquina de estados muy simple.

\begin{itemize}
	\item $q_0$ Estado Inicial: En espera
	\item $F=\{q_1\}$ Único estado final : Autentificado
	\item $\mathcal{A}=\{001,010,011,000\}$
\end{itemize}

En cuanto a la función de transición. Supongamos que nuestro sistema incorpora la siguiente directivas:

\begin{multicols}{2}
	\begin{itemize}
		\item Envío de contraseña incorrecta: $001$
		\item Envío de contraseña correcta: $010$
		\item Salir: $011$
		\item Error: $000$
	\end{itemize}
\end{multicols}

\vspace{1cm}

\begin{center}
	\begin{tikzpicture}[scale=0.2]
	\tikzstyle{every node}+=[inner sep=0pt]
	\draw [black] (37,-18.6) circle (3);
	\draw (37,-18.6) node {$Esp$};
	\draw [black] (37,-18.6) circle (2.4);
	\draw [black] (49.2,-18.6) circle (3);
	\draw (49.2,-18.6) node {$Act$};
	\draw [black] (49.2,-18.6) circle (2.4);
	\draw [black] (43.9,-31.8) circle (3);
	\draw (43.9,-31.8) node {$E$};
	\draw [black] (39.494,-16.958) arc (113.75402:66.24598:8.952);
	\fill [black] (39.49,-16.96) -- (40.43,-17.09) -- (40.02,-16.18);
	\draw (43.1,-15.7) node [above] {$011$};
	\draw [black] (46.998,-20.609) arc (-58.98762:-121.01238:7.566);
	\fill [black] (47,-20.61) -- (46.05,-20.59) -- (46.57,-21.45);
	\draw (43.1,-22.19) node [below] {$010$};
	\draw [black] (37,-15.5) -- (37,-15.6);
	\fill [black] (37,-15.6) -- (37.5,-14.8) -- (36.5,-14.8);
	\draw [black] (34.59,-16.833) arc (261.47443:-26.52557:2.25);
	\draw (31.9,-12.1) node [above] {$001$};
	\fill [black] (36.94,-15.61) -- (37.55,-14.9) -- (36.56,-14.75);
	\draw [black] (26.4,-18.6) -- (34,-18.6);
	\draw (25.9,-18.6) node [left] {$\epsilon$};
	\fill [black] (34,-18.6) -- (33.2,-18.1) -- (33.2,-19.1);
	\draw [black] (50.924,-21.032) arc (24.0977:-67.85012:7.646);
	\fill [black] (46.83,-31.24) -- (47.76,-31.4) -- (47.38,-30.47);
	\draw (51.79,-27.89) node [right] {${000,011}$};
	\draw [black] (40.919,-31.779) arc (-101.43921:-203.36619:7.786);
	\fill [black] (40.92,-31.78) -- (40.23,-31.13) -- (40.04,-32.11);
	\draw (34.88,-28.9) node [left] {${000,010}$};
	\draw [black] (49.35,-15.615) arc (204.86229:-83.13771:2.25);
	\draw (54.48,-12.2) node [above] {$010$};
	\fill [black] (51.66,-16.9) -- (52.6,-17.02) -- (52.18,-16.11);
	\end{tikzpicture}
\end{center}


Podríamos codificar los estados de $Esp$ (espera) y $Act$ (activo) como $000$ y $100$ con lo que tendríamos el estado de error ($E$) codificado como $101$. El autómata describe los posibles pasos en la autentificación de un usuario en un sistema, los estados de 'espera' y 'activo' serían finales dado que son estados 'correctos' en un posible sistema como este. El estado de error, describe justo eso, un comando que no se esperaba dado el estado de la comunicación en el que estábamos. Con este modelo podríamos construir un sistema de autentificación de usuarios usando 3 bits unicamente.


\subsection{La jerarquía de Chomsky}

El lenguaje que puede reconocer un autómata finito recibe el nombre de lenguaje regular, ya sea el autómata determinista, no-determinista o no-determinista con transiciones nulas, pues como comentábamos el paso de uno a otro se lleva a cabo mediante un procedimiento algorítmico. 

Un concepto íntimamente relacionado con el de lenguaje, como aquí lo hemos visto, es el concepto de gramática, a saber: las reglas de producción a seguir para crear una palabra del lenguaje. Distinguiremos símbolos terminales (escritos con letras mayúsculas) y símbolos terminales, notados con letras minúsculas. La gramática asociada a un lenguaje regular tiene la siguiente forma:


$$ A \rightarrow a B \quad \quad A \rightarrow a$$ 

Esto quiere decir que cada símbolo no terminal se intercambia por un símbolo terminal y otro no terminal a lo sumo y que, cada símbolo no terminal es resoluble, es decir, siempre se puede intercambiar por uno terminal. Pongamos un ejemplo: \\

Sea $\mathcal{L}=\{0,1\}^*$ el conjunto de todas las palabras que se pueden formar yuxtaponiendo 0's y 1's. Su gramática vendría dada por:


\begin{multicols}{2}
	\begin{itemize}
		\item $A \rightarrow 1$
		\item $A \rightarrow 0$
		\item $A \rightarrow 1A$
		\item $A \rightarrow 0A$  
	\end{itemize}
\end{multicols}


Esta gramática es muy sencilla debido al alto número de restricciones bajo las que se ha concebido, si eliminamos dichas restricciones aumentamos la expresividad del lenguaje, aumentando así la complejidad del modelo de computación que tendrá asociado, como el caso de los autómatas finitos para los lenguajes regulares.


Pues bien, la jerarquía de Chomsky es una estructura piramidal que refleja como aumenta la complejidad del lenguaje conforme eliminamos restricciones en la gramática:

\vspace{1cm}


\begin{center}
	\begin{tabular}{|c|c|c|c|}
		\hline 
		Gramática & Lenguaje  &Reglas de producción   & Autómata  \\ 
		\hline 
		Tipo 0	& recursivamente enumerable  & sin restricciones  & Máquina de Turing  \\ 
		\hline 
		Tipo 1	& dependiente del contexto  & $\alpha A \beta \rightarrow \alpha \gamma \beta$ & linealmente acotado  \\ 
		\hline 
		Tipo 2	& independiente del contexto  & $A \rightarrow \gamma $  & autómata con pila   \\ 
		\hline 
		Tipo 3	& regular  & (*) & autómata finito \\
		\hline 
	\end{tabular} 
\end{center}

\vspace{1cm}

Los autómatas con pila presentan una generalización de los autómatas finitos. Ya no tenemos una representación tipo grafo pero la manera de leer símbolos con la función $\delta(\cdot)$ es similar. Resaltamos que además del criterio de estados finales ( si teminamos de leer una palabra y estamos en un estado final la palabra es aceptada) los autómatas con pilas presentan un criterio de aceptación equivalente: el criterio de pila vacía; es decir: si al terminar de leer la palabra la pila del autómata está vacía se considerará una palabra válida. Esto se debe a que al leer un símbolo en la palabra en este tipo de autómatas podemos meter o sacar símbolos específicos de una pila ( lifo ).

Sin embargo, daremos un salto cualitativo importante e iremos directamente al modelo computacional más complejo de la jerarquía: Las máquinas de turing.



\subsection{La Máquina de Turing}





\newpage
	

\section{Historia Antigua de la Robótica}

\section{Historia Moderna de la Robótica}

%Introducción
\subsection{Introducción}
Acabamos de ver cómo fueron los comienzos de la historia de la robótica desde la época griega hasta los primeros robots industriales de la década de los 60. De aquí en adelante podemos considerar que entramos en la etapa más cercana a nuestros años con un hecho significativo que comenzaremos explicando y marca una diferencia con la etapa previa: la introducción concienzuda de la inteligencia artificial.

\vspace{10px}

Este salto de época viene marcado por la apertura del SRI (Stanford Research Institute) que comienza con la investigación en el campo de la inteligencia artificial tanto en la universidad de Stanford como en la de Edimburgo. Este hecho delimita la historia de la robótica pues es a partir de aquí cuando los investigadores comienzan a dar autonomía e inteligencia a los robots que se pretenden diseñar. Esta herramienta será fundamental en el entendimiento del desarrollo de los robots y las capacidades de los mismos.

\vspace{10px}

De aquí en adelante veremos como los robots han avanzado en una variedad inmensa de campos, incluyendo la movilidad, los brazos robóticos, los sistemas de reconocimiento del entorno y la interacción de los mismos. Actualmente a nadie se le escapa el hecho de que la robótica es una industria que se encuentra en pleno auge y que se prevee que continúe creciendo.

\vspace{10px}

En las secciones posteriores de este trabajo veremos como la robótica ha llegado hasta el momento más actual con aplicaciones como la industria militar, las investigaciones espaciales, las interfaces conversaciones o los robots humanoides, actualmente colocados en el ojo de mira de toda la prensa mundial.


%Desarrollo de los laboratorios de IA
\subsection{Inicios de la Inteligencia Artificial}
\subsubsection{SRI}
SRI o Stanford Research Institute fue una institución amparada bajo la Universidad de Stanford de propósito general en un inicio. Su fundación fue dificultosa y no fue hasta el tercer intento cuando se consiguió en el año 1946.

\vspace{10px}

El SRI hizo investigaciones en variados campos como agricultura, ejército o contaminación pero no fue hasta 1966 cuando el Centro de Inteligencia Artificial comenzó a funcionar bajo el mando de Charles Rosen. El primer proyecto realizado por el SRI en inteligencia artificial fue el Robot Shakey, el primer robot cuya intención no era sólo moverse, si no razonar. Este robot era capaz de analizar el entorno mediante una cámara y procesaba el lenguaje natural para recibir instrucciones. Desarrollando las habilidades de este robot en el desplazamiento por una habitación esquivando los obstáculos que se le imponían en la misma se desarrollaron algoritmos de Visión por Computador tales como filtros y convoluciones para detección de bordes de objetos a ser esquivados y además se desarrollaron algoritmos tan importantes como el A*, el algoritmo por excelencia de cálculo de posibles caminos.

\vspace{10px}

Shakey fue dado por finalizado en el año 1972, siendo ahora los esfuerzos del SRI dirigidos en la creación del Internet para suceder a ARPANET. La primera conexión que se produjo entre dos ordenadores usando el Internet como lo conocemos ahora se produjo en 1977 entre los laboratorios del SRI y la universidad de California UCLA.

\vspace{10px}

Tras estos grandes hitos el SRI se desprendió de la Universidad de Stanford para comenzar su andadura como una empresa, fundándose en ese momento la compañía SRI International, pero la Inteligencia Artificial ya había comenzado.

\subsubsection{Laboratorios de IA del MIT}
En la misma época el Instituto Tencnológico de Massachusetts o MIT abrió sus laboratorios de inteligencia artificial donde surgieron algunas de las figuras más reconocidas en el ámbito de la computación.

\vspace{10px}

El inicio de la investigación en el ámbito de la computación en el MIT comenzó con figuras tan reseñables como Shannon o Vannevar Bush, el ideólogo del analizador diferencial. En sus comienzos el laboratorio no empezó formándose como tal, si no que se creó un proyecto llamado MAC (Man and Computer) en el que estaban figuras tan importantes como John McCarthy el inventor de Lisp. En un principio el proyecto se dirigió intentando abarcar un amplio rango de temas, como son: la visión por computador, el movimiento mecánico y la manipulación de objetos a partir de máquinas y el manejo del lenguaje natural.

\vspace{10px}
De los despachos en los cuales se llevaban a cabo estas investigaciones surgieron otras figuras como Richard Stallman (creador del proyecto GNU) e invenciones como las máquinas Lisp (máquinas que ejecutaban muy eficientemente Lisp).


%Robotics Institute
\subsection{Robotics Institute}
La Inteligencia Artificial y el desarrollo de la electrónica fue dando paso cada vez con mas fuerza a una nueva rama de la computación: la robótica.

\vspace{10px}

Esta rama es completamente transversal, puesto que aglomera a todos los departamentos relacionados con la computación desde hardware a diseño de software pasando por la visión por computador o la Inteligencia Artificial. Sobre el año 1979 la Universidad Carnegie Mellon en Pittsburgh se dio cuenta del potencial desarrollo de la robótica en un futuro, por lo que decidió abrir un nuevo departamento en la misma que llamó Robotics Institute o Instituto de Robótica, siendo la primera Universidad en tener un departamento dedicado únicamente a la robótica.

\vspace{10px}

Dos robots que podemos destacar de este instituto fueron los vehículos Sandstorm y Highlander dos vehículos completamente autónomos diseñados para realizar un rallie por un desierto llegando entre puntos del mismo sin ninguna intervención humana. Actualmente los avances logrados en la implementación de estos dos vehículos están siendo implementados por las marcas de coches para intentar lograr cada vez una mayor independencia de los conductores de los mismos.

\vspace{10px}

Actualmente el Robotics Institute ya no es sólo un departamento de la Universidad Carnegie Mellon, sino una compañía en si misma con un presupuesto de unos 65 millones de dólares anuales. Algunos de los campos de trabajo en los que se centran son: robots relacionados con la agricultura, robots industriales, robots médicos, robots asistentes y robots relacionados con el transporte de personas y mercancías.


%WABOT
\subsection{El primer robot antropomórfico: WABOT}
Hasta este momento los robots habían sido proyectos que implementaban funcionalidades concretas tales como moverse, ver, intentar hablar, reemplazar a los humanos en algunas tareas repetitivas, etc.

\vspace{10px}

En la Universidad de Waseda en Japón, se creó el primer robot con forma antropomórfica llamado WABOT 1. Según los integrantes del proyecto, si quisiéramos que un robot nos ayudase como un asistente, sería conveniente que tuviera forma antropomórfica para que nos resultase más familiar, por lo que se plantearon crear un robot asistente inteligente con forma humana. En este proyecto se desarrollaron dos versiones del mismo robot el primero llamado WABOT 1 y tras este crearon WABOT 2.

\vspace{10px}

WABOT 1 fue creado entre los años 1970 y 1973 siendo el primer robot antropomórfico del mundo. El robot tenía control de sus brazos, incorporaba un sistema de visión y análisis del entorno y un sistema conversacional. Éste era capaz de medir su posición con respecto a objetos de la sala y medir las distancias hasta ellos. Además era capaz de andar y coger objetos. En aquel momento, al ser el robot capaz de hablar, se le realizó un test psicológico obteniendo como resultado que WABOT 1 se podría equiparar a un niño de un año de edad.

\vspace{10px}

Tras el éxito del proyecto el grupo de investigación pensó en desarrollar una nueva versión que llamaron WABOT 2 entre los años 1980 y 1984. Esta versión se convirtió en algo mucho menos útil y versátil que WABOT 1, ya que fue diseñado únicamente con el objetivo de que fuera capaz de tocar el piano. Según los propios investigadores esto era interesante pues las artes requieren pensar como un humano y destreza en el movimiento.

\vspace{10px}

Estos dos robots sentaron un precedente abriendo el campo de los robots antropomórficos, los cuales analizaremos en etapas posteriores más cercanas a nuestra época.


%Brazos robóticos
\subsection{Brazos Robóticos}
\subsubsection{Stanford Arm}

Los brazos robóticos han sido un gran pilar en el desarrollo de la tecnología por su gran utilidad. Actualmente rara es la fábrica que no emplea brazos robóticos para hacer la producción mucho más ágil e incluso segura para los operadores de la misma. Este camino fue iniciado en el año 1969 en la Universidad de Stanford por Victor Scheinman.

\vspace{10px}

El primer brazo robótico fue desarrollado bajo el proyecto Hand-Eye en los laboratorios de Inteligencia Artificial de la Universidad de Stanford. El Stanford Arm fue diseñado tras los intentos de hacer brazos operables mediante un humano desempeñados en dicha Universidad. Los intentos precedentes incluían modificaciones de brazos ortopédicos e incluso un modelo hidráulico que resultaba muy peligroso por la rapidez de sus movimientos. El Stanford Arm se creó sin tener una forma antropomórfica y con 6 grados de libertad manipulado de forma completamente eléctrica.

\vspace{10px}

El robot estaba controlado mediante cámaras y elementos parecidos a joysticks gracias a que el brazo llevaba incorporados potenciómetros y sensores que permitían el control del mismo.

\begin{figure}[!h]
	\centering
	\includegraphics[scale=0.3]{./EtapaModerna/Imagenes/stanford_arm.jpg}
	\caption{Stanford Arm}
	\label{fig:stanfordArm}
\end{figure}

\subsubsection{T3}

Tras los grandes avances del Stanford Arm la compañía Cincinnati Milacron Corp. decidió desarrollar en 1973 el primer brazo robótico pensado para ser incorporado dentro del mercado industrial. El encargado dentro de la compañía de elaborar el diseño del brazo fue Richard Hohn.

\vspace{10px}

Este brazo robótico, al igual que el Stanford Arm fue desarrollado con 6 grados de libertad permitiendo un amplio abanico de movimientos que el brazo podía hacer con bastante precisión.

\vspace{10px}

Este brazo ya no tenía que ser controlado directamente por un humano, si no que implementaba un módulo de decisiones y sensores capaz de saber si había completado una cierta tarea y tenía que realizar algún movimiento, por ejemplo coger un elemento de una cinta de transporte y moverlo.

\subsubsection{PUMA}

Tras el éxito de los diseños de Victor Scheinman en la Universidad de Stanford, la compañía General Motors se interesó en sus prototipos de brazos robóticos. Ante el incipiente mercado de estos robots las grandes compañías con un gran nivel de fábricas comenzaron a ver una utilidad en estos brazos, por lo que General Motors financió a Scheinman para que desarrollara el brazo robótico PUMA (Programmable Universal Machine for Assembly o Programmable Universal Manipulation Arm).

El brazo se comercializó en tres tamaños distintos con la intención de que unos levantasen más peso que otros. Todos los brazos tenían el mismo diseño y los mismos grados de libertad, por lo que los movimientos posibles eran iguales para todos los modelos. Los ángulos máximos de giro y rotación variaban entre modelos, ya que hay que pensar que estos fueron creados casi \textit{ad hoc} para las compañías. Además los modelos más grandes tenían unos movimientos mucho más lentos debido al peso que tenían que soportar y mover por el diseño tan grande que tenían para la época.

Este modelo fue fabricado por General Motors, Westinghouse Electric Corp., Staubli e incluso la división de robótica de Nokia. Fue un éxito fabricándose únicamente por Nokia unos 1500 brazos robóticos PUMA.

\subsubsection{MOGURA}

No solo se hicieron brazos robóticos convencionales siguiendo los diseños de Stanford, si no que también surgieron algunos intentos más novedosos como el brazo robótico MOGURA. Este brazo surge como un diseño del profesor Shigeo Hirose, del Instituto Tecnológico de Tokyo. El profesor Hirose estaba estudiando robots que imitaran el comportamiento de animales, como por ejemplo las serpientes, con las que ideó un robot llamado ACMVI que imitaba el comportamiento de las mismas. De las articulaciones que desarrolló para este robot surgió la idea de hacer un brazo con más movilidad aún al que llamó MOGURA entorno al año 1978. Este brazo no tuvo un éxito muy destacable, ya que los diseños anteriores aún funcionaban bien y los siguientes intentos como SCARA cumplieron su función muy notablemente.

\subsubsection{SCARA}
Un avance muy grande en el ámbito de los brazos robóticos fueron los brazos SCARA (Selective Compliance Assembly Robot Arm o Selective Compliance Articulated Robot Arm). Estos brazos rompían la estructura de los antiguos diseños haciendo que la movilidad de los mismos fuera muy superior con movimientos mucho más rápidos. Estos robots fueron conceptualizados por las compañías Japonesas Sankyo Seiki, Pentel y NEC y fueron elaborados bajo la supervisión de Hiroshi Makino, un profesor de la Universidad de Yamanashi.

El diseño de este tipo de brazos pasaba de un robot que se movía únicamente como los ejes cartesianos, es decir en movimientos rectos muy fijos, a un movimiento muy ágil con ejes paralelos. El brazo SCARA se movía igual que los anteriores cartesianos, es decir, de forma recta con respecto a los ejes X,Y y Z pero además incluía un ángulo de giro en el plano del eje Z de forma que podía rotar las piezas que cogía.

Este tipo de robots son extremadamente útiles en la manipulación entre elementos en líneas de trabajo de fábricas, ya que con ellos se hace muy sencillo transportar objetos entre distintas fases en el proceso de un objeto. Por ejemplo podemos imaginar una línea de trabajo en la que se hacen puertas para vehículos, este robot podría transportar la puerta ya finalizada de la línea de fabricación de puertas, rotarla convenientemente y desplazarla hacia la línea de ensamblaje en el chasis del coche, de forma que movería la puerta él solo y el operario sólo tendría que atornillarla o realizar el ensamblaje que corresponda. Estos robots, gracias al ángulo de giro en el eje Z, pueden servir por ejemplo para atornillar piezas en una placa que se les coloque en la línea de ensamblaje tales como por ejemplo los tornillos que fijan ciertos componentes a la placa base de un móvil u ordenador.

Estos brazos robóticos supusieron un avance muy notable en su creación siendo incluso utilizados actualmente. Hay que tener en cuenta que los brazos robóticos siguen aún en desarrollo y casi a diario obtenemos nuevas mejoras en los mismos, pero estos fueron los que sentaron precedentes en lo que actualmente conocemos y asociamos con brazos robóticos.


%Inicios los movimientos humanos
\subsection{Mejora en la movilidad de los Robots}
\subsubsection{Introducción}

Tras el desarrollo de robots primigenios con funcionalidad concreta pero fija se planteó la posibilidad de integrar movimiento por el terreno en los mismos para poder realizar nuevas tareas u otras de forma más eficiente. En este ámbito se invirtió para lograr avances en esquivar objetos del escenario o en sistemas para andar basados en animales cuadrúpedos o bípedos como nosotros. A continuación analizamos de forma cronológica estos hechos.

\subsubsection{RB5X de RB Robot Corporation}
Un inicio en la movilidad en los robots fue incorporarles ruedas u orugas como herramientas para su desplazamiento por el suelo. Estos tipos de movimientos eran básicos, tales como hacia delante, hacia atrás o hacia los lados. Un avance en el movimiento inteligente lo protagonizó el Stanford's Cart, el cual fue progamado con sensores para desplazarse de una punta a otra de un recinto esquivando los objetos a su paso. Este movimiento inteligente era realizado gracias a un sistema de Visión por Computador que incluía el mismo. Tras esto surgieron nuevos intentos como el RB5X, un robot de pequeñas dimensiones capaz de desplazarse y evitar obstáculos o, en caso de no poder evitarlos, capaz de reconstruir su trayectoria tras impactar con algún objeto.

\vspace{10px}

El robot se ideó de forma multipropósito, puesto que los módulos que integraba eran programables por lo que las empresas tomaron el diseño y lo adecuaron a las tareas que se querían. El robot tuvo variantes tales como asistentes personales a los cuales les podías dar órdenes como recoger el periódico y a través de un brazo robótico que incluía poder interactuar con el medio. Además se le desarrolló una interfaz oral básica para poder darles las órdenes mediante la voz. Además era capaz de aprender de su entorno.

\subsubsection{Phony Pony}

Tras los intentos de hacer robots que incorporasen ruedas u orugas la Universidad de California del Sur realizó el desarrollo del que se conoce como primer robot cuadrúpedo llamado Phony Pony. El diseño de este robot se realizó en 1968 (anterior a RB5X) por los profesores e investigadores Frank y McGhee.

\vspace{10px}

El robot intentaba imitar las articulaciones de los animáles cuadrúpedos que conocemos de forma que lo desarrollaron con dos articulaciones: la cadera y la rodilla. De esta forma el robot podía mover la pierna hacia delante y además flexionarla. La velocidad del movimiento de las patas era extremadamente lenta, ya que al igual que ocurría con el Stanford Arm (contemporáneo a Phony Pony) los elementos eléctricos empleados en el movimiento estaban muy limitados. El Phony Pony era capaz de imitar comportamientos tales como andar, arrastrarse, agacharse o trotar.

\vspace{10px}

El control de este robot se realizaba de forma remota gracias a la tecnología desarrollada ya en esa época. Internamente el robot estaba diseñado mediante patrones de movimiento, de forma que estaba implementado con un autómata finito que tomaba las transiciones del mismo tras recibir la entrada del mando de control.

\subsubsection{WAP1}
En primer paso que se dió entorno a la idea de realizar un robot bípedo tuvo lugar en 1969 en la Universidad de Waseda en Japón. El robot WAP1 (Waseda's antropomorphic pneumatically-activated pedipulators) fue diseñado por el doctor Ichiro Kato.

\vspace{10px}

El doctor Ichiro Kato trabajaba por aquel momento en el terreno médico y empezó a intentar desarrollar músculos artificiales para poder sustituir músculos defectuosos en sus pacientes. El estudio desarrrolló una serie de dispositivos que al hincharse adquirían las formas que ellos necesitaban para simular el comportamiento de un músculo. Estas investigaciones darían sus frutos plasmándose en el primer robot bípedo creado. Los músculos que permitían el movimiento de WAP1 estaban hechos de goma y se inflaban con actuadores neumáticos. Este primer robot no era capaz más que de andar muy lentamente controlando el flujo de aire que se llevaba hacia las piezas de goma. Estas técnicas hacían que WAP llegara a tardar mucho tiempo en completar un paso, además de que no estaba pensado para andar una distancia dando pasos de forma consecutiva.

\vspace{10px}

En los dos siguientes años el doctor Ichiro Kato mejoró el diseño del WAP1 creado el WAP2 y el WAP3. La segunda versión de estos robots tuvo una mejora en la fuerza que los músculos eran capaces de aplicar sobre la estructura del mismo, pudiendo ahora levantar más peso y tener una estructura más robusta. La última versión fue desarrollada en 1971 y en esta el robot ya era capaz de rotar su movimiento, por lo que ya no tenían que ser movimientos rectos y, además, era capaz de superar pequeños obstáculos tales como ligeras inclinaciones o incluso escaleras de pequeño tamaño. Para poder desarrollar todo esto no sólo fueron necesarios los músculos artificiales del doctor Kato, si no además sus estudios en el control y mejora de la postura con lo que fueron capaces de controlar el centro de gravedad del robot, cosa completamente imprescindible para caminar sobre dos piernas.

\subsubsection{WL-9DR}
Tras los avances del doctor Ichiro Kato se intentó agilizar el movimiento de estos robots, puesto que el tiempo que había que esperar hasta que el WAP era capaz de avanzar un paso hacía que no fuera práctico su uso, ya que los robots que empleaban ruedas por ejemplo eran mucho más ágiles. Entre los años 1979 y 1980 la Universidad de Waseda desarrolló el WL-9DR bajo la dirección del doctor Kato, al igual que con sus predecesores los WAP.

\vspace{10px}

El funcionamiento de esta máquina también era mediante actuadores hidráulicos que operaban más rápidamente que los diseñados para WAP. Así mismo el control de este robot se hacía mediante un procesador de 16 bit conectado al robot mediante cables. El WL-9RD era capaz de dar un paso en 10 segundos, además de realizar ciclos completos andando, es decir, encadenar pasos para avanzar desde el punto de partida a un punto meta que se le fijaba. La velocidad con la que avanzaba era tan lenta que sus creadores no podían podían denominar al robot como dinámico, por lo que se le conoce como el primer robot bípedo cuasi-dinámico.

\vspace{10px}

El robot tenía un sistema de pesos que mantenía el equilibrio del mismo como se requiriera, por ejemplo si el robot estaba estático se centraba el punto de apoyo y se mantenía la postura tanto por la gran base rectangular de sus pies como por los pesos centrados. Si se quería dar un paso hacia delante el robot alzaba alguna de las piernas y se cambiaba el centro de los pesos un poco hacia delante, de forma que el robot se inclinara ligeramente para seguir con el siguiente paso. Para poder andar solamente 0.5 metros el robot requería sobre una docena de pasos llevándole entorno a un minuto completar dicha distancia.

\subsubsection{Aquarobot}
Tras los robots anteriormente mencionados (que sentaron precedentes en el tema de caminar) se siguieron desarrollando las técnicas que se empleaban en el equilibrio y las partes móviles de los mismos tal y como veremos en las siguientes secciones. Entre estos robots surgieron algunos proyectos que intentaron ir un poco más allá como el Aquarobot cuyo objetivo era ser capaz de andar bajo el agua.

\vspace{10px}

En el instituto de investigaciones de puertos de Japón se pensó en que sería útil que los operarios que tenían que trabajar bajo el agua dispusieran de facilidades para hacer la tarea más rápida y sencilla. Una de los trabajos que más tiempo llevaban en el puerto eran los de inspección bajo el agua para certificar el estado de las estructuras o diagnosticar algún tipo de fallo que se debía solventar. Por ello en el año 1984 se desarrolló la primera versión del Aquarobot, llegando a tener este hasta tres versiones diferentes haciendo mejoras sobre lo ya propuesto.

\vspace{10px}

El diseño básico del robot era una estructura con 6 patas capaz de andar bajo el agua. Los motores que empleaban eran de corriente continua (no neumáticos) colocados en cada una de las patas que iban selladas de forma estanca para impedir la entrada del agua. Además el robot era controlado por un pequeño procesador incluido en el mismo al que se le conectaban cables para comunicarlo con una computadora o un mando de control. El Aquarobot disponía asimismo de una cámara integrada que permitía ver en tiempo real lo que el robot observaba, de forma que los técnicos podían visualizar el estado del puerto sin necesidad de bajar ellos al agua.

\vspace{10px}

Cada pata tenía 3 grados de libertad con los que el robot podía andar y rotar sobre sí mismo. El peso aproximado de este robot era de unos 857 kilogramos en su primera versión y de unos 280 en la tercera. El sistema con el que se manipulaba al robot y los patrones de movimiento que se le definían en su propio procesador estaba programados en C++. El tiempo en el que el robot era capaz de responder a las órdenes que se le daban era de unos 50 milisegundos, gracias a la conexión que se tenía mediante cable.

\subsubsection{Honda y el gran avance}
La compañía Honda Motors, normalmente conocida por la fabricación de coches, tuvo un papel muy importante en el desarrollo de los robots antropomórficos y bípedos con sus series de robots, desde la serie E experimental, pasando por la P hasta los más que conocidos robots ASIMO que perduran actualmente.

\vspace{10px}

La serie de robots Honda E contó con 7 modelos, desde el 0 hasta el 6 desarrollados entre los años 1986 y 1993. Esta serie experimental comenzó en el mismo punto que el robot WAP, desarrollando un sistema que podía caminar sobre dos piernas. Estos robots, desde la versión 0, mejoraron enormemente los tiempos conseguidos por sus predecesores necesitando, ``únicamente'', 5 segundos para dar un paso completo. Este modelo contaba con 6 grados de libertad y con las articulaciones similares a las de una pierna humana, una cadera, la rodilla y el tobillo. En la versión 1 se mejoró la velocidad de movimiento pero se aumentó el peso haciéndo que fuera notablemente más grande que su versión anterior, además se le añadió un sistema mucho más complejo de articulaciones que le permitían tener 12 grados de libertad. En las versiones 2,3,4 y 5 se mejoró tanto el peso como la velocidad al andar, logrando que el robot anduviera de forma autónoma y dinámica llegando a los 4.7 kilómetros por hora. Cabe destacar que el diseño acabó con una cabeza muy grande y un peso de 150 kilogramos (comparados con los 16.5 de la versión 0). En la útltima versión no se modificó tanto el diseño y la velocidad al andar, si no que se centraron en que el robot fuera capaz de controlar el equilibrio de forma inteligente y fuese capaz de esquivar obstáculos e incluso no andar sobre un suelo liso.

\vspace{10px}

Tras el gran avance que lograron los ingenieros de Honda en cuanto a la capacidad de desplazamientos de la serie E, decidieron tomar todos esos avances y añadir más elementos para que pareciera cada vez más una persona. Tras la serie E vino la P con la intención de tomar lo obtenido de forma experimental en la anterior serie para continuar su camino hacia los robots antropomórficos. Estos robots fueron desarrollados entre los años 1993 y 2000. El primer modelo de la serie P supuso una transición en la cual se le añadieron brazos que, aunque no eran capaces de llevar una carga, le servían al robot para interactuar con el medio con acciones tales como abrir puertas o encender y apagar interruptores. En el modelo P2 Honda quiso marcar un precedente en la historia de la robótica haciendo que se pudiera desplazar andando, subir escaleras, empujar objetos e incluso llevar carga en las manos. El robot fue el primero en el que no se utilizaron cables para la comunicación con el mismo, para lo cual llevaba una batería incluida que permitía que trabajase alrededor de unos 15 minutos. La opinión pública se sorprendió enormemente, ya que este fue el primer robot conocido que tuvo un movimiento muy similar al humano. Tras el segundo modelo honda sacó el P3, cuyo objetivo con respecto a la versión anterior era reducir el peso, cosa que consiguieron ya que se pasó de 210 kilogramos a 130 sin variar las dimensiones del mismo más que en la profundidad del cuerpo del robot. El diseño además adquirió una semejanza mayor aún a los humanos con respecto al P2 y se mejoró la eficiencia de uso en la batería lo que permitió una autonomía de 25 minutos en lugar de los 15 minutos del P2. Por último se sacó el Honda P4, el que fue el último en la serie P de robots. Las dimensiones no variaron con respecto al P3 pero si el peso que se redujo a 80 kilogramos. Además este robot fue el que consiguió un movimiento más completo de la serie P con 34 grados de libertad contando brazos y piernas pero no se profundizó especialmente en esta versión puesto que los robots ASIMO estaban ya en desarrollo y Honda decidió acabar esta serie para dedicar todos sus esfuerzos en la nueva gama de robots.

\vspace{10px}

La serie de robots ASIMO o (Advanced Step in Innovative Mobility) fueron la continuación de la serie P de Honda en cuanto a la creación de un robot humanoide. Los robots ASIMO tuvieron diferentes revisiones, esta vez no divididas en diferentes versiones. La primera versión se sacó a la luz en el año 2000 para después tener nuevas mejoras en los años 2002, 2005, 2007 y 2011. Los ASIMO comenzaron teniendo una altura de 120 centímetros y 54 kilogramos de peso, un gran avance comparando los pesos y alturas con los obtenidos en la serie P que llegaban a pesar 210 kilogramos. ASIMO mejoró en la velocidad en la que andaba pasando de 1.6 kilómetros por hora a 2.7 kilómetros por hora. En sus primeros modelos ASIMO no era capaz de correr pero en el modelo de 2004 la funcionalidad se añadió comenzando a correr a 3 kilómetros por hora hasta los 9 de la versión de 2011. La batería que incorporaban los Honda P fue reutilizada en las primeras versiones de los ASIMO pero ésta cambió a una batería de ión de litio desde el 2004 para llegar a dar una hora completa de batería en uso mixto (andando y corriendo). Además los grados de libertad que obtuvo la versión de 2011 fueron 57, un número muy superior a los 26 del modelo del año 2000 que permitía movimientos no sólo de brazos y piernas si no también movimientos completos de los dedos de la mano con lo que la última versión es capaz de hacer gestos con la misma y movimientos complejos parecidos a los que podemos realizar con nuestra mano. Además en la última versión se le añadió un sistema de voz con el que las personas nos podemos comunicar con ASIMO para pedirle tareas específicas. En su última versión el robot ASIMO es capaz de reconocer el entorno que le rodea mediante cámaras y sensores de visión incorporados en su casco con los cuales ASIMO no sólo es capaz de interactuar con las personas por voz, si no que también es capaz de analizar los movimientos y comportamientos de las mismas para actuar en función de ello. Además en cuanto a la interfaz oral no sólo responde mediante altavoces para continuar la conversación, si no que además también gesticula para dar una interacción más humana.

Con los robots ASIMO Honda completó una era en cuanto a los movimientos en la robótica al conseguir un movimiento al caminar natural y rápido y una interacción con el medio a través de las manos y brazos muy extensa. Tras esto los robots antropomórficos han seguido avanzando pero todos utilizando en gran medida la senda que Honda marcó con su series E, P y ASIMO.


%iRobot
\subsection{iRobot}
En el camino de la creación de robots hasta este momento se habían diseñado pensando siempre en su aplicación empresarial o industrial. ASIMO pudo ser el primero en el que la gente se pudiera replantear su utilidad dentro de una casa para cumplir tareas sencillas, pero debido a su alto precio (incluso no fueron comercializados más allá de hacer demostraciones para empresas) no eran accesibles a nadie o casi nadie. En este punto surgió la compañía iRobot Corporation, una compañía fundada por tres ingenieros del MIT: Rodney Brooks, Colin Angle y Helen Greiner quienes venían de trabajar y estudiar en el Laboratorio de Inteligencia Artificial del MIT.

iRobot cambió un poco la perspectiva de las personas al hacer que se incluyeran en casa robots de un precio asequible cuyas aplicaciones reales eran verdaderamente útiles (no como un asistente ASIMO en una casa).

La compaía fue fundada en 1990, época por la cual trabajaban con robots aplicados a la tecnología militar, como fue su robot PackBot usado en las guerra de Iraq y Afghanistan como apoyo a las tropas. El robot era útil puesto que incorporaba un sistema que le permitía superar obstáculos tales como piedras, escaleras, huecos, etc gracias a las orugas con las que se desplazaba. iRobot además no sólo tuvo contratos con la industria militar si no que, además, mantuvo relaciones con la NASA desarrollando robots para ellos también.

Tras los contratos mencionados anteriormente 


%Estado actual de la robótica
\subsection{State of Art de la Robótica}
Hasta ahora hemos visto cómo la robótica dió sus primeros pasos tras la concepción de los laboratorios de Inteligencia Artificial por todo el mundo, con lo que se produjo un boom en las técnicas y el desarrollo de la robótica. No se le escapará a nadie que actualmente a la cabeza de la robótica podemos encontrar ciertas secciones o categorías que se están desarrollando por encima del resto ya sea por su utilidad, por lo vistosos que son o por el simple hecho de buscar un desarrollo en dicho campo. Estas categorías podrían ser: los robots usados en el espacio o expediciones espaciales, los robots usados en la industria militar, los robots diseñados mediante la IA para algún tipo de tarea concreta o los robots que imitan cualidades humanas.

A continuación trataremos de discutir el actual estado de los robots en estas categorías.

\subsubsection{NASA}
Dentro del panorama aeroespacial la agencia que lidera actualmente el desarrollo en la exploración fuera de la Tierra es la NASA. Gracias a esto la agencia espacial estadounidense ha desarrollado un programa de robótica muy amplio y robusto, con robots que van desde la asistencia en viajes espaciales hasta exploración del espacio, telescopios o robots usados en exploraciones planetarias.

Entre los robots diseñados por la NASA no sólo hay robots pensados para la exploración de planetas, también se han ideados robots de asistencia para los astronautas. En este grupo de robots podemos meter a la serie Robonaut con las dos versiones desarrolladas hasta el momento. Esta gama de robots están pensados para la asistencia tanto dentro de la nave espacial como fuera de la misma, por ejemplo en tareas de reparación o acoplamiento de módulos. El primer modelo, el Robonaut 1 o R1 nunca llegó al espacio, si no se que empleó como un primer prototipo para estudio y desarrollo. El robot consistía de un torso con forma humana y con la parte de abajo intercambiable, de esta forma se podía adecuar el mismo para la exploración de planetas o colocarle módulos como el Zero-G Leg, pensado para poder engancharse a las barras exteriores que se le suelen colocar a los módulos, naves y estaciones espaciales para facilitar el desplazamiento de los astronautas por el exterior de las mismas (en esta misma línea la NASA tiene un concepto de robot llamado Spidernaut). De esta manera se demostraba que este robot, si se diseñara de forma adecuada, sería útil en misiones de reparación en el exterior de las naves o de asistencia. En su segunda versión el robot mejoró todas sus capacidades y el diseño se convirtió realmente en algo funcional siendo lanzado en 2011 a la ISS (Estación Espacial Internacional). El primer modelo que se envió a la estación no era capaz de moverse pero sí de llevar cosas y mover sus brazos de forma que era útil para sostener cosas como herramientas o moverlas de un sitio a otro con la rotación del torso. Tras ese paso se enviaron a la ISS piernas y módulos extra al robot de forma que ahora ya es capaz de moverse dentro de la estación y es capaz de ver y analizar su entorno. Por el momento no se ha conseguido presurizar el cuerpo lo suficiente como para poder emplearlo en misiones fuera de la estación. Actualmente esta es la línea de desarrollo del robot tanto para misiones fuera de la estación como para misiones de exploración del espacio profundo.

En este mismo sentido la compañía canadiense MDA, en su aportación a la ISS, diseñó el robot Dextre con una funcionalidad muy parecida a la ideada para Robonaut. El robot consiste de un cuerpo de enormes dimensiones con dos brazos robóticos capaces de moverse muy rápido. El objetivo de este robot es que sea capaz de realizar tareas para las que antes se requería la salida de alguno de los astronautas de la ISS al exterior de la misma. Este hecho se ha conseguido de tal forma que ni siquiera se tiene por qué operar desde la ISS, si no que se puede realizar el manejo del mismo desde la Tierra. De esta forma, tal y como ocurrió en su primera misión, se pueden realizar tareas de mantenimiento sin necesidad de que los miembros de la ISS estén disponibles o incluso aunque estos estén dormidos. El robot tiene un cuerpo que mide unos 3.5 metros de longitus y dos brazos de 3.5 metros cada uno pesando en total más de 1600 kilogramos.

La NASA también tiene robots que aún no han sido empleados en misiones reales pero sí se sigue avanzando en su desarrollo, como es el caso del RASSOR (Regolith Advanced Surface Systems Operations Robot). Este robot está diseñado para operaciones de construcción y alisamiento del terrreno, extracción de agua, eliminar el hielo de una zona concreta, retirar arena, etc. El RASSOR actualmente está en fase de pruebas, pero se le espera un buen futuro por su mínimo coste de producción y por la versatilidad del mismo, puesto que sería capaz de trabajar unas 16 horas por día durante muchos años. El robot además, por su diseño, está pensado para superar los obstáculos que se le presenten y para poder corregir su posición incluso si vuelca. Para las tareas que se planean que pueda realizar el robot incorpora dos ruedas como las de una trituradora industrial (una en la parte delantera y otra en la trasera) capaces de ser movidas de forma independiente.

En último lugar en los robots creados por la NASA tenemos el que actualmente es la punta de lanza de la compañía: el Curiosity. Este modelo es un vehículo terrestre con componentes robóticos que aglomera todos los conocimientos obtenidos por la NASA en cuanto a robótica. Este robot pertenece a la serie de los Rovers y actualmente está cumpliendo su misión de exploración de la superficie marciana. El Curiosity fue lanzado hacia Marte en el año 2011 y aterrizó en el año 2012 tras un viaje de unos 8 meses. El robot tiene un complejo sistema de análisis que permite que opere de forma autónoma, ya que en este caso la distancia del operador (la Tierra) y el vehículo es tan grande que llevaría unos 14 minutos mandar una orden desde las oficinas de la NASA y que el Curiosity la recibiera. El robot pesa unos 900 kilogramos, midiendo 2.9 metros de largo, 2.7 metros de ancho y 2.2 metros en altura. En cuanto a capacidad de computación se incorporan chips de IBM diseñados especificamente para el mismo, ya que al estar en Marte y haber tenido que soportar el viaje por el espacio se requiere que todos los dispositivos tales como microprocesadores, memoria RAM y memoria flash vayan convenientemente aisladas de la radiación, pues podrían verse alterados los datos o incluso dejar de funcionar. Para alimentar los sistemas eléctricos, el Curiosity incorpora un generador de energía por radioisótopos con lo que lleva incluído 4.8 kilogramos de dióxido de plutonio 238 que le dan una autonomía mínima de 14 años desde su despliegue. En cuanto a el equipo robótico el Curiosity incorpora un sistema de suspensiones y ruedas pensadas para poder superar fácilmente los obstáculos del terreno marciano tales como roca o arena. Todo el sistema de movilidad del robot está diseñado para que, mediante un sistema complejo de cámaras llamadas Hazcams (Hazard Cameras) se detecte el terreno del rover en un entorno circular de 3 metros a su alrededor de forma que sea capaz de predecir si un obstáculo será o no superable para él. La velocidad del vehículo no es nada elevada siendo su velocidad media 30 metros por hora, pudiendo llegar al pico de 90 metros por hora.

En cuanto al equipamiento técnico para obtener datos de Marte el Curiosity incorpora los siguientes elementos:

\begin{enumerate}
  \item MastCam (Mast Camera): está compuesta por dos cámaras independientes capaces de sacar fotos true-color y vídeo a 10 fotogramas por segundo. Cada cámara incorpora una memoria de 8 GB capaces de almacenar 5500 imágenes sin compresión.
  \item ChemCam (Chemistry and Camera complex): este sistema de cámaras incorpora un sistema de análisis por rayos X, espectrómetros y cámaras que analizan muestras del suelo marciano para obtener su composición.
  \item SAM (Sample Analysis at Mars): el objetivo de este módulo es analizar compuestos orgánicos y gases tanto de la atmósfera como de muestras sólidas. Incorpora un espectrómetro de masas, un cromatógrafo de gases y un espectrómetro laser.
  \item DRT (Dust Removal Tool): esta herramienta sirve al robot para limpiar el polvo de las rocas o terreno sobre el que se queire obtener una muestra.
  \item RAD (Radiation assessent detector): esta herramienta fue diseñada para medir la radiación a la que se expone una nave en un viaje interplanetario con la intención de saber qué requisitos deben tener las naves espaciales para poder llevar personas en distancias tan largas. Además, actualmente, este módulo se está usando para saber los niveles de la radiación en el propio planeta.
  \item DAN (Dynamic Albedo of Neutrons): este tubo de neutrones se utiliza para poder obtener medidas de hidrógeo, agua o hielo en la superficie de Marte.
  \item MARDI (Mars Descent Imager): la función que cumplió este módulo fue tomar un buen número de imágenes durante el aterrizaje para poder hacer un mapeado del terreno alrededor del Curiosity, de forma que se tuviera una información de qué terreno rodeaba al robot y localizar su posición de aterrizaje.
  \item Brazo robótico: el Curiosity incorpora un único brazo robótico de 2.1 metros de largo y 30 kilogramos de peso. El brazo sirve al robot para poder tomar las muestras y mover elementos como por ejemplo pinzas para tomar muestras, un taladro de pequeñas dimensiones para tomar muestras sólidas, el DRT, etc.
\end{enumerate}

Estos avances han sido cruciales en el desarrollo de la ciencia y la tecnología tanto por las altas inversiones de dinero que fomentan la investigación como por los experimentos realizados gracias a la tecnología desarrollada por la NASA. Actualmente se siguen liderando propuestas innovadoras tales como Mars 2020 o el reciente proyecto (ya aterrizado en Marte) Mars InSight.

\subsubsection{Robots militares}
En el ámbito de la industria militar y el equipo empleado en la guerra se ha realizado un avance enorme en el tipo de tecnología que se involucra en la robótica. En esta área no se persiguen interfaces conversacionales complejas ni robots con forma humana (por el momento) si no que está más enfocada en vehículos no tripulados que se manejen por sí solos o teleoperados de forma que la guerra, en su gran mayoría no sea un combate cuerpo a cuerpo para minimizar las pérdidas humanas desde el lado del atacante. En este sentido la robótica comenzó a formar parte desde la Srgunda Guerra Mundial y la posterior Guerra Fría, época en la que la inversión en el material militar aumentó considerablemente tanto en Europa en general como en Estados Unidos y Rusia en particular, por lo que estos países jugarán una baza importante en el desarrollo de la tecnología al ser los implicados directamente en los conflictos bélicos.

Actualmente hay muchos robos en desarrollo que serán importantes en un futuro en las misiones a las que se los destine, pero podemos destacar desde un punto de vista de la relevancia y actualidad de los mismos a los siguientes modelos:

\begin{enumerate}
  \item Daksh de DRDO: este robot no está pensado para el uso directo en el combate atacando objetivos, si no que se ideó con el propósito de que fuese capaz de inutilizar material explosivo antes de que las tropas pasaran por un lugar concreto. El robot funciona mediante baterías y es capaz de operar por él solo levantando objetos y remolcando vehículos para despejar el camino. Además incorpora un mecanismo de desactivación de bombas que consiste en inyectar agua a alta presión para inutilizar el material. Todas estas labores las puede hacer el robot por sí mismo, por lo que está diseñado para analizar el terreno a su alrededor, ser capaz de esquivar o superar obstáculos y organizar las tareas que debe llevar a cabo para desactivar bombas. A pesar de todo esto, si el equipo de combate necesita operarlo manualmente, incorpora un sistema de comunicación que permite su operación remota.
  \item Elbit Hermes 450: este vehículo aéreo entra dentro de la categoría de los llamados UAV (Unmanned Aerial Vehicle). El objetivo de este vehículo es la exploración del terreno y detección de posibles objetivos y peligros. Actualmente muchos países han adquirido este UAV que es capaz de operar durante 17 horas seguidas. El robot tiene un alto grado de independencia de los operadores, puesto que se puede pedir que explore una zona concreta y él lo hará todo autónomamente, aún así también se incorpora un sistema de comunicación por satélite para operarlo manualmente si se requiere. Aunque en un principio se diseñó como vehículo exploratorio, actualmente hay algunos países como Estados Unidos que han incorporado misiles a este UAV, de forma que también se podría emplear en el derribo de objetivos.
  \item Goalkeeper CIWS (Close-in Weapon System): el Goalkeeper es un sistema autónomo de defensa de embarcaciones de corto alcance. El sistema incorpora una metralleta GAU-8 de calibre 30 capaz de, mediante un sistema de reconocimiento de objetivos, fijar la mayor amenaza que tiene el barco y comenzar el ataque hacia la misma. Este propósito es resuelto mediante dos radares con los que se pueden obtener los objetivos a atacar, calcular la distancia y posición de los mismos y poder detectar el que él considera mas amenzador. Además el sistema es capaz de detectar la velocidad a la que viaja el enemigo y predecir la trayectoria, cosa que es imprescindible para poder acertar en el blanco.
  \item Guardium: también se requieren en la industria robots que operen de forma autónoma en el ámbito terrestre en la batalla. Actualmente este modelo ha sido empleado por Israel en la franja de Gaza. El robot puede operar de forma autónoma fijándole objetivos o teleoperado con lo que un militar podría dirigir el robot hacia el objetivo manualmente. El robot va equipado con una armadura externa para resistir los impactos de bala y armamento tanto de supresión del enemigo (no letal) como armamento pesado de guerra. Incorpora cámaras infrarrojas, radares, micrófonos de alta sensibilidad y un conjunto de sensores capaces de detectar de dónde vienen los disparos para poder protegerse. Si el robot entra en modo autónomo analiza sus alrededores y toma deciciones sobre las localizaciones a las que se debe dirigir identificando al enemigo. El robot tiene una autonomía de 103 horas de uso continuado y se suele utilizar en rondas de vigilancia para protección de las personas en zonas de conflicto. Las cámaras no sólo duran las 103 horas comentadas, si no que además incorporan un sistema de emergencia que podrían hacer que las cámaras funcionasen 24 horas extra por si son necesarias en alguna misión de vigilancia.
\end{enumerate}


\normalsize


% Bibliografía.
%-----------------------------------------------------------------
\onecolumn
\bibliography{EtapaModerna}
\bibliographystyle{plain}
\nocite{*}

\end{document}
