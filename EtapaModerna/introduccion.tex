Acabamos de ver cómo fueron los comienzos de la historia de la robótica desde la época griega hasta los primeros robots industriales de la década de los 60. De aquí en adelante podemos considerar que entramos en la etapa más cercana a nuestros años con un hecho significativo que comenzaremos explicando y marca una diferencia con la etapa previa: la introducción concienzuda de la inteligencia artificial.

Este salto de época viene marcado por la apertura del SRI (Stanford Research Institute) que comienza con la investigación en el campo de la inteligencia artificial tanto en la universidad de Stanford como en la de Edimburgo. Este hecho delimita la historia de la robótica pues es a partir de aquí cuando los investigadores comienzan a dar autonomía e inteligencia a los robots que se pretenden diseñar. Esta herramienta será fundamental en el entendimiento del desarrollo de los robots y las capacidades de los mismos.

Estos avances en la robótica han afectado diversos campos como la medicina, la industria militar, la domótica ... Estos avances sobre todo han sido gracias a la inteligencia artificial y al estudio de la electrónica más avanzada. A continuación veremos el avance histórico reciente de este campo.
