Acabamos de ver cómo fueron los comienzos de la historia de la robótica desde la época griega hasta los primeros robots industriales de la década de los 60. De aquí en adelante podemos considerar que entramos en la etapa más cercana a nuestros años con un hecho significativo que comenzaremos explicando y marca una diferencia con la etapa previa: la introducción concienzuda de la inteligencia artificial.

Este salto de época viene marcado por la apertura del SRI (Stanford Research Institute) que comienza con la investigación en el campo de la inteligencia artificial tanto en la universidad de Stanford como en la de Edimburgo. Este hecho delimita la historia de la robótica pues es a partir de aquí cuando los investigadores comienzan a dar autonomía e inteligencia a los robots que se pretenden diseñar. Esta herramienta será fundamental en el entendimiento del desarrollo de los robots y las capacidades de los mismos.

De aquí en adelante veremos como los robots han avanzado en una variedad inmensa de campos, incluyendo la movilidad, los brazos robóticos, los sistemas de reconocimiento del entorno y la interacción de los mismos. Actualmente a nadie se le escapa el hecho de que la robótica es una industria que se encuentra en pleno auge y que se prevee que conntinúe creciendo.

En las secciones posteriores de este trabajo veremos como la robótica ha llegado hasta el momento más actual con aplicaciones como la industria militar, las investigaciones espaciales, las interfaces conversaciones o los robots humanoides, actualmente colocados en el ojo de mira de toda la prensa mundial.
