Hasta ahora hemos visto cómo la robótica dió sus primeros pasos tras la concepción de los laboratorios de Inteligencia Artificial por todo el mundo, con lo que se produjo un boom en las técnicas y el desarrollo de la robótica. No se le escapará a nadie que actualmente a la cabeza de la robótica podemos encontrar ciertas secciones o categorías que se están desarrollando por encima del resto ya sea por su utilidad, por lo vistosos que son o por el simple hecho de buscar un desarrollo en dicho campo. Estas categorías podrían ser: los robots usados en el espacio o expediciones espaciales, los robots usados en la industria militar, los robots diseñados mediante la IA para algún tipo de tarea concreta o los robots que imitan cualidades humanas.

A continuación trataremos de discutir el actual estado de los robots en estas categorías.

\subsubsection{NASA}
Dentro del panorama aeroespacial la agencia que lidera actualmente el desarrollo en la exploración fuera de la Tierra es la NASA. Gracias a esto la agencia espacial estadounidense ha desarrollado un programa de robótica muy amplio y robusto, con robots que van desde la asistencia en viajes espaciales hasta exploración del espacio, telescopios o robots usados en exploraciones planetarias.

Entre los robots diseñados por la NASA no sólo hay robots pensados para la exploración de planetas, también se han ideados robots de asistencia para los astronautas. En este grupo de robots podemos meter a la serie Robonaut con las dos versiones desarrolladas hasta el momento. Esta gama de robots están pensados para la asistencia tanto dentro de la nave espacial como fuera de la misma, por ejemplo en tareas de reparación o acoplamiento de módulos. El primer modelo, el Robonaut 1 o R1 nunca llegó al espacio, si no se que empleó como un primer prototipo para estudio y desarrollo. El robot consistía de un torso con forma humana y con la parte de abajo intercambiable, de esta forma se podía adecuar el mismo para la exploración de planetas o colocarle módulos como el Zero-G Leg, pensado para poder engancharse a las barras exteriores que se le suelen colocar a los módulos, naves y estaciones espaciales para facilitar el desplazamiento de los astronautas por el exterior de las mismas (en esta misma línea la NASA tiene un concepto de robot llamado Spidernaut). De esta manera se demostraba que este robot, si se diseñara de forma adecuada, sería útil en misiones de reparación en el exterior de las naves o de asistencia. En su segunda versión el robot mejoró todas sus capacidades y el diseño se convirtió realmente en algo funcional siendo lanzado en 2011 a la ISS (Estación Espacial Internacional). El primer modelo que se envió a la estación no era capaz de moverse pero sí de llevar cosas y mover sus brazos de forma que era útil para sostener cosas como herramientas o moverlas de un sitio a otro con la rotación del torso. Tras ese paso se enviaron a la ISS piernas y módulos extra al robot de forma que ahora ya es capaz de moverse dentro de la estación y es capaz de ver y analizar su entorno. Por el momento no se ha conseguido presurizar el cuerpo lo suficiente como para poder emplearlo en misiones fuera de la estación. Actualmente esta es la línea de desarrollo del robot tanto para misiones fuera de la estación como para misiones de exploración del espacio profundo.

En este mismo sentido la compañía canadiense MDA, en su aportación a la ISS, diseñó el robot Dextre con una funcionalidad muy parecida a la ideada para Robonaut. El robot consiste de un cuerpo de enormes dimensiones con dos brazos robóticos capaces de moverse muy rápido. El objetivo de este robot es que sea capaz de realizar tareas para las que antes se requería la salida de alguno de los astronautas de la ISS al exterior de la misma. Este hecho se ha conseguido de tal forma que ni siquiera se tiene por qué operar desde la ISS, si no que se puede realizar el manejo del mismo desde la Tierra. De esta forma, tal y como ocurrió en su primera misión, se pueden realizar tareas de mantenimiento sin necesidad de que los miembros de la ISS estén disponibles o incluso aunque estos estén dormidos. El robot tiene un cuerpo que mide unos 3.5 metros de longitus y dos brazos de 3.5 metros cada uno pesando en total más de 1600 kilogramos.

La NASA también tiene robots que aún no han sido empleados en misiones reales pero sí se sigue avanzando en su desarrollo, como es el caso del RASSOR (Regolith Advanced Surface Systems Operations Robot). Este robot está diseñado para operaciones de construcción y alisamiento del terrreno, extracción de agua, eliminar el hielo de una zona concreta, retirar arena, etc. El RASSOR actualmente está en fase de pruebas, pero se le espera un buen futuro por su mínimo coste de producción y por la versatilidad del mismo, puesto que sería capaz de trabajar unas 16 horas por día durante muchos años. El robot además, por su diseño, está pensado para superar los obstáculos que se le presenten y para poder corregir su posición incluso si vuelca. Para las tareas que se planean que pueda realizar el robot incorpora dos ruedas como las de una trituradora industrial (una en la parte delantera y otra en la trasera) capaces de ser movidas de forma independiente.
