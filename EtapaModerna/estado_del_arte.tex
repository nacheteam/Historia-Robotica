Hasta ahora hemos visto cómo la robótica dió sus primeros pasos tras la concepción de los laboratorios de Inteligencia Artificial por todo el mundo, con lo que se produjo un boom en las técnicas y el desarrollo de la robótica. No se le escapará a nadie que actualmente a la cabeza de la robótica podemos encontrar ciertas secciones o categorías que se están desarrollando por encima del resto ya sea por su utilidad, por lo vistosos que son o por el simple hecho de buscar un desarrollo en dicho campo. Estas categorías podrían ser: los robots usados en el espacio o expediciones espaciales, los robots usados en la industria militar, los robots diseñados mediante la IA para algún tipo de tarea concreta o los robots que imitan cualidades humanas.

A continuación trataremos de discutir el actual estado de los robots en estas categorías.

\subsubsection{NASA}
Dentro del panorama aeroespacial la agencia que lidera actualmente el desarrollo en la exploración fuera de la Tierra es la NASA. Gracias a esto la agencia espacial estadounidense ha desarrollado un programa de robótica muy amplio y robusto, con robots que van desde la asistencia en viajes espaciales hasta exploración del espacio, telescopios o robots usados en exploraciones planetarias.

Entre los robots diseñados por la NASA no sólo hay robots pensados para la exploración de planetas, también se han ideados robots de asistencia para los astronautas. En este grupo de robots podemos meter a la serie Robonaut con las dos versiones desarrolladas hasta el momento. Esta gama de robots están pensados para la asistencia tanto dentro de la nave espacial como fuera de la misma, por ejemplo en tareas de reparación o acoplamiento de módulos. El primer modelo, el Robonaut 1 o R1 nunca llegó al espacio, si no se que empleó como un primer prototipo para estudio y desarrollo. El robot consistía de un torso con forma humana y con la parte de abajo intercambiable, de esta forma se podía adecuar el mismo para la exploración de planetas o colocarle módulos como el Zero-G Leg, pensado para poder engancharse a las barras exteriores que se le suelen colocar a los módulos, naves y estaciones espaciales para facilitar el desplazamiento de los astronautas por el exterior de las mismas (en esta misma línea la NASA tiene un concepto de robot llamado Spidernaut). De esta manera se demostraba que este robot, si se diseñara de forma adecuada, sería útil en misiones de reparación en el exterior de las naves o de asistencia. En su segunda versión el robot mejoró todas sus capacidades y el diseño se convirtió realmente en algo funcional siendo lanzado en 2011 a la ISS (Estación Espacial Internacional). El primer modelo que se envió a la estación no era capaz de moverse pero sí de llevar cosas y mover sus brazos de forma que era útil para sostener cosas como herramientas o moverlas de un sitio a otro con la rotación del torso. Tras ese paso se enviaron a la ISS piernas y módulos extra al robot de forma que ahora ya es capaz de moverse dentro de la estación y es capaz de ver y analizar su entorno. Por el momento no se ha conseguido presurizar el cuerpo lo suficiente como para poder emplearlo en misiones fuera de la estación. Actualmente esta es la línea de desarrollo del robot tanto para misiones fuera de la estación como para misiones de exploración del espacio profundo.

En este mismo sentido la compañía canadiense MDA, en su aportación a la ISS, diseñó el robot Dextre con una funcionalidad muy parecida a la ideada para Robonaut. El robot consiste de un cuerpo de enormes dimensiones con dos brazos robóticos capaces de moverse muy rápido. El objetivo de este robot es que sea capaz de realizar tareas para las que antes se requería la salida de alguno de los astronautas de la ISS al exterior de la misma. Este hecho se ha conseguido de tal forma que ni siquiera se tiene por qué operar desde la ISS, si no que se puede realizar el manejo del mismo desde la Tierra. De esta forma, tal y como ocurrió en su primera misión, se pueden realizar tareas de mantenimiento sin necesidad de que los miembros de la ISS estén disponibles o incluso aunque estos estén dormidos. El robot tiene un cuerpo que mide unos 3.5 metros de longitus y dos brazos de 3.5 metros cada uno pesando en total más de 1600 kilogramos.

La NASA también tiene robots que aún no han sido empleados en misiones reales pero sí se sigue avanzando en su desarrollo, como es el caso del RASSOR (Regolith Advanced Surface Systems Operations Robot). Este robot está diseñado para operaciones de construcción y alisamiento del terrreno, extracción de agua, eliminar el hielo de una zona concreta, retirar arena, etc. El RASSOR actualmente está en fase de pruebas, pero se le espera un buen futuro por su mínimo coste de producción y por la versatilidad del mismo, puesto que sería capaz de trabajar unas 16 horas por día durante muchos años. El robot además, por su diseño, está pensado para superar los obstáculos que se le presenten y para poder corregir su posición incluso si vuelca. Para las tareas que se planean que pueda realizar el robot incorpora dos ruedas como las de una trituradora industrial (una en la parte delantera y otra en la trasera) capaces de ser movidas de forma independiente.

En último lugar en los robots creados por la NASA tenemos el que actualmente es la punta de lanza de la compañía: el Curiosity. Este modelo es un vehículo terrestre con componentes robóticos que aglomera todos los conocimientos obtenidos por la NASA en cuanto a robótica. Este robot pertenece a la serie de los Rovers y actualmente está cumpliendo su misión de exploración de la superficie marciana. El Curiosity fue lanzado hacia Marte en el año 2011 y aterrizó en el año 2012 tras un viaje de unos 8 meses. El robot tiene un complejo sistema de análisis que permite que opere de forma autónoma, ya que en este caso la distancia del operador (la Tierra) y el vehículo es tan grande que llevaría unos 14 minutos mandar una orden desde las oficinas de la NASA y que el Curiosity la recibiera. El robot pesa unos 900 kilogramos, midiendo 2.9 metros de largo, 2.7 metros de ancho y 2.2 metros en altura. En cuanto a capacidad de computación se incorporan chips de IBM diseñados especificamente para el mismo, ya que al estar en Marte y haber tenido que soportar el viaje por el espacio se requiere que todos los dispositivos tales como microprocesadores, memoria RAM y memoria flash vayan convenientemente aisladas de la radiación, pues podrían verse alterados los datos o incluso dejar de funcionar. Para alimentar los sistemas eléctricos, el Curiosity incorpora un generador de energía por radioisótopos con lo que lleva incluído 4.8 kilogramos de dióxido de plutonio 238 que le dan una autonomía mínima de 14 años desde su despliegue. En cuanto a el equipo robótico el Curiosity incorpora un sistema de suspensiones y ruedas pensadas para poder superar fácilmente los obstáculos del terreno marciano tales como roca o arena. Todo el sistema de movilidad del robot está diseñado para que, mediante un sistema complejo de cámaras llamadas Hazcams (Hazard Cameras) se detecte el terreno del rover en un entorno circular de 3 metros a su alrededor de forma que sea capaz de predecir si un obstáculo será o no superable para él. La velocidad del vehículo no es nada elevada siendo su velocidad media 30 metros por hora, pudiendo llegar al pico de 90 metros por hora.

En cuanto al equipamiento técnico para obtener datos de Marte el Curiosity incorpora los siguientes elementos:

\begin{enumerate}
  \item MastCam (Mast Camera): está compuesta por dos cámaras independientes capaces de sacar fotos true-color y vídeo a 10 fotogramas por segundo. Cada cámara incorpora una memoria de 8 GB capaces de almacenar 5500 imágenes sin compresión.
  \item ChemCam (Chemistry and Camera complex): este sistema de cámaras incorpora un sistema de análisis por rayos X, espectrómetros y cámaras que analizan muestras del suelo marciano para obtener su composición.
  \item SAM (Sample Analysis at Mars): el objetivo de este módulo es analizar compuestos orgánicos y gases tanto de la atmósfera como de muestras sólidas. Incorpora un espectrómetro de masas, un cromatógrafo de gases y un espectrómetro laser.
  \item DRT (Dust Removal Tool): esta herramienta sirve al robot para limpiar el polvo de las rocas o terreno sobre el que se queire obtener una muestra.
  \item RAD (Radiation assessent detector): esta herramienta fue diseñada para medir la radiación a la que se expone una nave en un viaje interplanetario con la intención de saber qué requisitos deben tener las naves espaciales para poder llevar personas en distancias tan largas. Además, actualmente, este módulo se está usando para saber los niveles de la radiación en el propio planeta.
  \item DAN (Dynamic Albedo of Neutrons): este tubo de neutrones se utiliza para poder obtener medidas de hidrógeo, agua o hielo en la superficie de Marte.
  \item MARDI (Mars Descent Imager): la función que cumplió este módulo fue tomar un buen número de imágenes durante el aterrizaje para poder hacer un mapeado del terreno alrededor del Curiosity, de forma que se tuviera una información de qué terreno rodeaba al robot y localizar su posición de aterrizaje.
  \item Brazo robótico: el Curiosity incorpora un único brazo robótico de 2.1 metros de largo y 30 kilogramos de peso. El brazo sirve al robot para poder tomar las muestras y mover elementos como por ejemplo pinzas para tomar muestras, un taladro de pequeñas dimensiones para tomar muestras sólidas, el DRT, etc.
\end{enumerate}

Estos avances han sido cruciales en el desarrollo de la ciencia y la tecnología tanto por las altas inversiones de dinero que fomentan la investigación como por los experimentos realizados gracias a la tecnología desarrollada por la NASA. Actualmente se siguen liderando propuestas innovadoras tales como Mars 2020 o el reciente proyecto (ya aterrizado en Marte) Mars InSight.

\subsubsection{Robots militares}
En el ámbito de la industria militar y el equipo empleado en la guerra se ha realizado un avance enorme en el tipo de tecnología que se involucra en la robótica. En esta área no se persiguen interfaces conversacionales complejas ni robots con forma humana (por el momento) si no que está más enfocada en vehículos no tripulados que se manejen por sí solos o teleoperados de forma que la guerra, en su gran mayoría no sea un combate cuerpo a cuerpo para minimizar las pérdidas humanas desde el lado del atacante. En este sentido la robótica comenzó a formar parte desde la Srgunda Guerra Mundial y la posterior Guerra Fría, época en la que la inversión en el material militar aumentó considerablemente tanto en Europa en general como en Estados Unidos y Rusia en particular, por lo que estos países jugarán una baza importante en el desarrollo de la tecnología al ser los implicados directamente en los conflictos bélicos.

Actualmente hay muchos robos en desarrollo que serán importantes en un futuro en las misiones a las que se los destine, pero podemos destacar desde un punto de vista de la relevancia y actualidad de los mismos a los siguientes modelos:

\begin{enumerate}
  \item Daksh de DRDO: este robot no está pensado para el uso directo en el combate atacando objetivos, si no que se ideó con el propósito de que fuese capaz de inutilizar material explosivo antes de que las tropas pasaran por un lugar concreto. El robot funciona mediante baterías y es capaz de operar por él solo levantando objetos y remolcando vehículos para despejar el camino. Además incorpora un mecanismo de desactivación de bombas que consiste en inyectar agua a alta presión para inutilizar el material. Todas estas labores las puede hacer el robot por sí mismo, por lo que está diseñado para analizar el terreno a su alrededor, ser capaz de esquivar o superar obstáculos y organizar las tareas que debe llevar a cabo para desactivar bombas. A pesar de todo esto, si el equipo de combate necesita operarlo manualmente, incorpora un sistema de comunicación que permite su operación remota.
  \item Elbit Hermes 450: este vehículo aéreo entra dentro de la categoría de los llamados UAV (Unmanned Aerial Vehicle). El objetivo de este vehículo es la exploración del terreno y detección de posibles objetivos y peligros. Actualmente muchos países han adquirido este UAV que es capaz de operar durante 17 horas seguidas. El robot tiene un alto grado de independencia de los operadores, puesto que se puede pedir que explore una zona concreta y él lo hará todo autónomamente, aún así también se incorpora un sistema de comunicación por satélite para operarlo manualmente si se requiere. Aunque en un principio se diseñó como vehículo exploratorio, actualmente hay algunos países como Estados Unidos que han incorporado misiles a este UAV, de forma que también se podría emplear en el derribo de objetivos. Dentro de esta misma categoría podríamos meter a los MQ-9 de Estados Unidos cuya misión es la misma que la que cumple el Elbit Hermes 450.
  \item Goalkeeper CIWS (Close-in Weapon System): el Goalkeeper es un sistema autónomo de defensa de embarcaciones de corto alcance. El sistema incorpora una metralleta GAU-8 de calibre 30 capaz de, mediante un sistema de reconocimiento de objetivos, fijar la mayor amenaza que tiene el barco y comenzar el ataque hacia la misma. Este propósito es resuelto mediante dos radares con los que se pueden obtener los objetivos a atacar, calcular la distancia y posición de los mismos y poder detectar el que él considera mas amenzador. Además el sistema es capaz de detectar la velocidad a la que viaja el enemigo y predecir la trayectoria, cosa que es imprescindible para poder acertar en el blanco.
  \item Guardium: también se requieren en la industria robots que operen de forma autónoma en el ámbito terrestre en la batalla. Actualmente este modelo ha sido empleado por Israel en la franja de Gaza. El robot puede operar de forma autónoma fijándole objetivos o teleoperado con lo que un militar podría dirigir el robot hacia el objetivo manualmente. El robot va equipado con una armadura externa para resistir los impactos de bala y armamento tanto de supresión del enemigo (no letal) como armamento pesado de guerra. Incorpora cámaras infrarrojas, radares, micrófonos de alta sensibilidad y un conjunto de sensores capaces de detectar de dónde vienen los disparos para poder protegerse. Si el robot entra en modo autónomo analiza sus alrededores y toma deciciones sobre las localizaciones a las que se debe dirigir identificando al enemigo. El robot tiene una autonomía de 103 horas de uso continuado y se suele utilizar en rondas de vigilancia para protección de las personas en zonas de conflicto. Las cámaras no sólo duran las 103 horas comentadas, si no que además incorporan un sistema de emergencia que podrían hacer que las cámaras funcionasen 24 horas extra por si son necesarias en alguna misión de vigilancia.
  \item TALON: El TALON es un robot terrestre no tripulado usado en la guerra de Bosnia con la intención de reducir las amenazas por explosivos a las tropas que se desplazan de forma terrestre. El robot está diseñado de forma modular, de esta manera cada país puede adaptarlo mediante la incorporación o retirada de sensores del mismo. El robot está pensado para poder desplazar los elementos de su entorno con la intención de despejar el camino e investigarlo. Para ello tiene una capacidad de carga de 45 kilogramos, una capacidad de tomar 77.11 kilogramos con su pinza, remolcar 340 kilogramos y la capacidad de levantar hasta 9 kilogramos con su brazo. El robot incorpora micrófonos para escuchar su entorno y altavoces para que sirva de elemento de comunicación de ser necesario. Adicionalmente TALON puede incorporar sensores de detección de gases, químicos, radiacción, temperatura, GPS y brújula. Si se requiere que el mismo tenga una mayor fuerza y capacidad de carga con su brazo se le puede añadir un eje de rotación mejor que ayudaría a llevar por ejemplo armas de largo alcance. En el terreno de la desactivación de explosivos también pueden incorporarse al mismo sensores de rayos X, una escopeta, herramientas de corte de cables, etc. El robot es operado mendiante un portátil y joysticks que controlan tanto el movimiento como el brazo y los elementos añadidos al mismo. PAra ello el robot incorpora baterías recargables y un paquete de baterías extra no recargables para casos de emergencia. Además la comunicación se realiza mediante un sistema de cámaras rotatorias, cámaras térmicas y micrófonos de forma que el operador tenga una información muy detallada de lo que rodea al TALON.
  \item SGR-A1: Este robot es una ametralladora automática pensada para disparar sin la necesidad de un humano. Este tipo de ametralladoras están pensadas para la defensa continua de una zona conflictiva, actualmente este modelo está siendo empleado en la zona desmilitarizada que une Corea del Norte y Corea del Sur (una de las fronteras más conflictivas del mundo). El proyecto está clasificado por su reciente creación, con lo que no tenemos información muy precisa de su funcionamiento. Lo que si sabemos es que funciona mediante cámaras infrarrojos, cámaras térmicas y cámaras normales para detectar figuras humanas como posibles blancos, la distancia que hay hasta ellas y las posibles coberturas que tiene la misma. Además está pensada para operar tanto de día como de noche. Esta arma tiene dos maneras de disparar: una en la que no se necesita ninguna intervención humana y otra en la que se necesita que reciba un comando de voz para que comience a disparar. La primera de las opciones ha creado un revuelo importante entorno a los derechos humanos y la atribución de las muertes, ya que esta arma está considerada la primera en matar de forma autónoma sin la orden expresa de un operador.
\end{enumerate}

Como podemos ver en el ámbito de la industria militar se han desarrollado enormemente las capacidades de movilidad de los robots tanto aéreamente como de forma terrestre con lo que ello conlleva en los sistemas de cámaras y comunicación con los mismos.

\subsubsection{Avances de la IA}
Como hemos podido ver la Inteligencia Artificial ha acompañado a todos los modelos que hemos ido viendo a lo largo de la historia moderna de la robótica por lo que, lo presentado en esta sección, no va a estar alejado del resto de los apartados pero sí podemos destacar algunas de las intervenciones de la IA en algunos campos concretos en los que se han logrado grandes avances. 
