La Inteligencia Artificial y el desarrollo de la electrónica fue dando paso cada vez con mas fuerza a una nueva rama de la computación: la robótica.

\vspace{10px}

Esta rama es completamente transversal, puesto que aglomera a todos los departamentos relacionados con la computación desde hardware a diseño de software pasando por la visión por computador o la Inteligencia Artificial. Sobre el año 1979 la Universidad Carnegie Mellon en Pittsburgh se dió cuenta del potencial desarrollo de la robótica en un futuro, por lo que decidió abrir un nuevo departamento en la misma que llamó Robotics Institute o Instituto de Robótica, siendo la primera Universidad en tener un departamento dedicado únicamente a la robótica.

\vspace{10px}

Dos robots que podemos destacar de este instituto fueron los vehículos Sandstorm y Highlander dos vehículos completamenta autónomos diseñados para relizar un rallie por un desierto llegando entre puntos del mismo sin ninguna intervención humana. Actualmente los avances logrados en la implementación de estos dos vehículos están siendo implementados en las marcas de coches actuales para intentar lograr cada vez una mayor independización de los conductores de los mismos.

\vspace{10px}

Actualmente el Robotics Institute ya no es sólo un departamento de la Universidad Carnegie Mellon, sino una compañía en si misma con un presupuesto de unos 65 millones de dólares anuales. Actualmente algunos de los campos de trabajo en los que se centran son: robots relacionados con la agricultura, robots industriales, robots médicos, robots asistentes y robots relacionados con el transporte de personas y mercancías.
