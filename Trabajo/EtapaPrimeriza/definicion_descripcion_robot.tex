La realidad es que, lo que hoy día consideramos , no se asemeja a lo que en sus inicios se llamaba robot. Pero sí que existen características generales que nos permiten llamar de la misma forma a los robot que se empezaban a desarrollar desde la antigua Grecia, y que naturalmente eran muy rudimentarios, y a los robot que conocemos hoy día y que ayudan notablemente a facilitar la vida diaria de las personas.\\

Si nos limitamos a definiciones como pueden ser de la R.A.E. se dice que un robot es \textit{"Máquina o ingenio electrónico programable, capaz de manipular objetos y realizar operaciones antes reservadas solo a las personas."} Además, un robot también debe de contar con las siguientes características:

\begin{itemize}
\item \textbf{Multifuncionalidad:} Capaz de llevar a cabo varias tareas.
\item \textbf{Programable:} Capaz de modificar su funcionalidad sin un excesivo costo.
\item \textbf{Alto Grado de Autonomía:} Capaz de ejecutar su tarea sin la intervención de un humano.
\end{itemize}

Pero nada más leer esta definición nos damos cuenta que no es válida para los robots que se describirán a continuación ya que alrededor del año 400 a.C. no se podía hablar de máquinas programables y mucho menos electrónicas. La realidad es que el concepto de robots tal y como se conoce hoy en día, no se desarrolló hasta 1948 de la mano de George Devol. Más adelante se explicará con mayor profundidad la importancia de este en la historia de la robótica.\\

A modo de curiosidad, la palabra \textit{robot} tiene su origen en la obra teatral \textit{Robots Universales Rossum} dirigida por un dramaturgo checo que se estrenó en 1920. Es por tanto, que no fue hasta 1920 cuando comenzó a llamarse \textit{robot} a todo lo que vamos a describir en próximos apartados.
