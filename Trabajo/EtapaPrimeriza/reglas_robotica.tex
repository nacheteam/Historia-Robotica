Isaac Asimov nacido en 1920 en la ciudad de Petrovichi, RSFS de Rusia. Con apenas tres años de edad su familia se trasladó a los EE.UU. y fue allí donde desarrolló su vida familiar, académica y laboral. A modo de curiosidad, señalar que debido a su precocidad intelectual sus padres decidieron falsificar su fecha de nacimiento para ingresar antes en un colegio neoyorquino. Una vez en la universidad se licenció e Químicas, Ciencias y Artes, Bioquímica y se doctoró en Filosofía. A lo largo de su vida tuvo dos esposas pero en ningún caso descendencia. Finalmente en 1992 fallece en Nueva York a causa del sida que había contraído en una transfusión sanguínea.\\

Durante su vida, Asimov destacó como escritor de ciencia-ficción y de divulgación científica. Fue en su libro \textit{"Yo robot"} donde este aventuró la relación que tendríamos los humanos con los robots, pero claro dichos robots no garantizaban la seguridad y prevención total de los riesgos. Es por ello que Asimov estableció las tres siguientes reglas de la robótica:

\begin{enumerate}
\item Un robot no hará daño a un ser humano o, por inacción, permitirá que un ser humano sufra daño.
\item Un robot debe cumplir las órdenes dadas por los seres humanos, a excepción de aquellas que entrasen en conflicto con la primera ley
\item Un robot debe proteger su propia existencia en la medida en que esta protección no entre en conflicto con la primera o con la segunda ley
\end{enumerate}

Señalar que Isaac Asimov era un adelantado a su tiempo y es que apostaba por la divulgación científica a través de la red. Es por ello que en 1988 aventuró la existencia de una plataforma como Wikipedia 13 años antes de que esta existiera. Además afirmaba la existencia en el futuro de plataformas como Skype, FaceTime o el desarrollo de coches autónomos, inventos que hoy en día son una realidad.
