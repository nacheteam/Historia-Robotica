Cuando nos referimos a Sistemas Mecánicos Simples\ref{sms} se hace referencia a aquellos formados principalmente por componentes, dispositivos o elementos que tienen como función transformar o transmitir el movimiento desde las fuentes que los generan al transforman distintos tipos de energía.

A continuación se describen los principales Sistemas Mecánicos Simples:

\begin{itemize}

\item \textbf{La Rueda:} Pieza mecánica circular que rota alrededor de un eje. Existen diferente tipos:

\begin{itemize}
\item El Rodillo: Se trata de un cilindro con un diámetro ancho y con una destacada longitud, que al igual que lo hace la rueda, gira alrededor de un eje.

\item Tren de Rodadura
\item Rueda Dentada: Se trata de una rueda cuyo perímetro está totalmente cubierto de dientes. Su ventaja recae en proporcionar movimiento circular mediante el contacto de los dientes entre dos ruedas.

\item Polea Fija: Se trata de una polea en el que la polea se encuentra fija en la parte superior.

\item Polea Móvil: Se trata de una polea conectada a una cuerda en cuyos extremos está, por un lado, anclada a un punto fijo y en otro a un extremo móvil.

\item Polipasto: Conjunto de poleas en la cual una queda fija y la otra tiene movilidad.


\end{itemize}

\item \textbf{La Palanca}: Existen diferente tipos.

\begin{itemize}
\item Palanca de Primer Grado: Se trata de una palanca donde el punto de apoyo se sitúa entre la Potencia y la Resistencia. Un ejemplo claro de este tipo de palanca podría ser unas tijeras o una balanza.

\item Palanca de Segundo Grado: Se trata de una palanca en la cual la Resistencia se sitúa entre el punto de apoyo y la fuerza. Un ejemplo de este tipo de palanca podría ser un cascanueces o una carretilla de obra.

\item Palanca de Tercer Grado: Se trata de una palanca en la cual la fuerza se sitúa entre la resistencia y el punto de apoyo. Un ejemplo de este tipo de palanca podría ser un martillo o una caña de pescar.

\end{itemize}

\item \textbf{Plano Inclinado}: Existen diferente tipos.

\begin{itemize}
\item La Rampa: Superficie plana que forma un ángulo agudo con la superficie.

\item Cuña: Se trata de un prisma triangular que actúa como un plano inclinado móvil.

\item Sistema Tornillo-Tuerca: Sistema en el que un tornillo rota en el interior de una tuerca. Se utiliza para unir de forma no permanente dos elemento.

\item Tirafondo: Tipo de tornillo que tiene una cabeza diseñada para ejercer el giro en esta mediante la ayuda de una herramienta.

\end{itemize}

\end{itemize}
