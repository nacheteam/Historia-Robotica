Leonardo da Vinci es un artista renacentista conocido principalmente por sus aportaciones en la pintura con son \textit{La Mona Lisa} o \textit{La Última Cena}. Pero este también era arquitecto, botánico, escritor, científico, filósofo, ingeniero e inventor entre otras cosas. Es por ellos que a continuación se va a hablar sobre su importante aportación al ámbito de la robótica.\\

En torno al año 1495, Leonardo da Vinci diseñó un autómata con forma de humano que se conoce como \textit{El robot de Leonardo}. Este tenía forma de guerrero medieval y el objetivo de su construcción es la aplicación práctica del estudio sobre el canon de proporciones humanas que se refleja en su obra \textit{El Hombre de Vitruvio}. No se tienen restos en la actualidad de aquel autómata pero si que existen bocetos que nos permiten confirmar su existencia además de que han permitido probar que funcionaba correctamente. Gracias a dichos bocetos se volvió a reconstruir en el año 2007.\\

Más tarde, en 1515 Leonardo creó un león mecánico que sirvió como regalo a la monarquía francesa, concretamente al rey Francisco I, cuando la comunidad florentina hizo una nueva alianza con Francia. Se trataba de un león de tamaño real y con una gran melena rizada. Este león es capaz de caminar, mover la cabeza, la cola y la boca y lanzar pétalos de flores de lis de uno de sus compartimentos, símbolo este de la monarquía francesa. Este, al igual que ocurrió con el \textit{El robot de Leonardo} ha sido reconstruido recientemente gracias a los bocetos encontrados de la época.
