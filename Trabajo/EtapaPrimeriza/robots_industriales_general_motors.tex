Por último, como fin a la etapa primeriza de la historia de la robótica vamos a hablar sobre \textbf{General Motors}, empresa norteamericana que destaca como fabricante mundial de vehículos. En 1957 se ofrece a esta empresa el conocido como Controlador Digital Modular, que está compuesto por máquinas de estado secuencial y que trabajan con procesadores centrales con desplazamiento de bits, que es el principio de lo que hoy conocemos como sistemas de control automático.\\

Pero sin duda el hito de \textit{General Motors} fue incorporar el primer robot industrial de la historia a sus instalaciones de Trenton, Nueva Jersey, en el año 1960. Este robot era un brazo robótico de 1800 kg de peso y que tenía la función de mover y colocar piezas metálicas de gran tamaño a grandes temperaturas. Señalar que este robot fue comprado a la empresa \textit{\textbf{Unimation}}, que como se ha comentado en apartados anteriores fue la primera empresa robótica de la historia fundada por  George Devol y el empresario  Joseph F. Engelberger.

Es en el 1978 cuando el robot \textit{PUMA}, desarrollado por la empresa de Devol y Engelberger, fue instalado en \textit{General Motors} para realizar tareas de montaje.
