\textbf{George Devol} es conocido como el creador del primer robot industrial además de cofundador de la primera empresa robótica de la historia llamada \textit{\textbf{Unimation}} junto al empresario  Joseph F. Engelberger.

Este tenía el objetivo de crear una máquina que fuese fácilmente manejable y que trabajara de forma automática. Fue en el año 1948 cuando Devol patentó el prototipo de lo que más tarde se convertiría el primer robot de uso industrial. Este robot era un brazo que tenía la función de mover artículos de gran tamaño y gran peso en el ámbito industrial. Además era capaz de mover piezas a altísimas temperaturas y de hasta 225 kilos de peso.\\

Señalar que Devol y  Engelberger formaban un tándem perfecto, uno aportaba sus cualidades creativa y el otro las suyas empresariales. Esto les permitió fundar lo que fue la primera empresa robótica de la historia además de conseguir un contrato con la empresa \textit{General Motors} para instalar el brazo robótico en su fábrica de Nueva Jersey. Poco después, en 1968, firmó un contrato similar con la empresa japonesa de \textit{Kawasaki}. Señalar que en esa época las dos potencias industriales mundiales eran Japón y Estados Unidos y que por el contrario en Europa este tema estaba más estancado.

Más tarde, en 1978, Devol consigue mejorar este brazo robótico haciéndolo programable y multiarticulado, lo que permitía colocar las piezas en cualquier posición que se quisiese siempre que estuviese a su alcance. Este fue llamado \textbf{\textit{PUMA}} (\textit{Programmable Universal Machine for Assembly}).

Señalar que George Devol murió en 2011 a los 99 años de edad y que no se le puede negar la gran aportación a la industria que ha realizado ya que puso la primera piedra de la industrias robotizadas de la actualidad.
