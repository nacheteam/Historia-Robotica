Hemos hablado de lo complicado que resulta dar una definición completa de estos conceptos, pues aunque todos sepamos qué es una maquina por extensión, las cualidades que comparten los objetos que llamamos máquinas pueden llegar a ser muy escasas.

\subsubsection{Álgebras de Boole y sistemas combinacionales}

En el siglo XIX se da otro gran avance cualitativo sobre el entendimiento de qué cosas son máquinas, alejándonos de la definición clásica de los mecanicistas y entrando en el campo de la lógica.

\vspace{10px}

Las álgebras de Boole son la base teórica de los sistemas combinacionales (S.C.). Un sistema combinacional es una máquina (según hemos definido) cuya salida es únicamente función de su entrada, es decir, no presenta memoria sobre ningún estado pasado. Este será un aspecto fundamental en el avance de las 'máquinas', su memoria. La habilidad de poder tener presente situaciones anteriores para deliberar en la ejecución actual, y en qué medida se pueda hacer esto, marcará toda una escala de complejidad.

\vspace{10px}


La realización de esta idea se ve con gran claridad en los circuitos lógicos, a partir de operaciones simples ( como son la negación, la disyunción y la conjunción ) podemos llegar a expresar teóricamente cuál sería el funcionamiento de un sistema lo suficientemente complejo como para realizar operaciones artiméticas a mayor velocidad que un humano.

\vspace{10px}

Queremos destacar que aunque estos sistemas se consideran típicamente como digitales, hechos a base de transistores recibiendo impulsos eléctricos, no tienen una diferencia cualitativa importante en cuanto a capacidad de computo o expresividad con un sistema mecánico clásico.

\vspace{10px}

En 1642 el filósofo y matemático francés Blaise Pascal inventó una calculadora mecánica, \textbf{La pascalina}. Aunque es evidente que hay una gran diferencia entre una calculadora mecánica y una digital, ambas pueden conceptualizarse como sistemas combinacionales; ya que ninguna de las dos tiene memoria sobre ejecuciones anteriores (estados) y el sistema, sean cuales sean sus actuadores, produce una salida como función única de una entrada.
