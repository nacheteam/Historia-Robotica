Por último queremos destacar la delgada linea que encontramos entre muchos de los conceptos de los que hemos hablado. Es decir: los extremos están controlados, queda claro que \textbf{Deep Blue} (1996), presentaba un comportamiento inteligente; también queda claro que una tostadora (por muy útil que sea) no lo presenta; pero, ¿Qué pasa si empezamos a añadirle funcionalidades a la tostadora? Si por ejemplo fuese capaz de saber cuando las tostadas están hechas y nunca las quemase ya entraría en consideración como agente reactivo. Las siguientes preguntas quedan sin responder: ¿Cómo de inteligente es un comportamiento dado?, ¿A partir de qué momento sería la tostadora inteligente?, ¿Queda mucho para vivir en el utópico mundo en el que no haya que rascar 'lo quemado' de las tostadas?

\vspace{10px}

Estas preguntas son más serias de lo que parecen en un mundo cada vez más informatizado. Desde luego llamamos inteligentes a los móviles, son MTU claramente, un concepto alejado de la implementación del protocolo de comunicación que supusieron en un principio..

\vspace{10px}

Citando a Turing (1950): \textbf{'Solo podemos prever el futuro inmediato, pero de lo que cabe duda es de que hay mucho por hacer'}.
