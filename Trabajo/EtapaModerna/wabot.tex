Hasta este momento los robots habían sido proyectos que implementaban funcionalidades concretas tales como moverse, ver, intentar hablar, reemplazar a los humanos en algunas tareas repetitivas, etc.

\vspace{10px}

En la Universidad de Waseda en Japón, se creó el primer robot con forma antropomórfica llamado WABOT 1. Según los integrantes del proyecto, si quisiéramos que un robot nos ayudase como un asistente, sería conveniente que tuviera forma antropomórfica para que nos resultase más familiar, por lo que se plantearon crear un robot asistente inteligente con forma humana. En este proyecto se desarrollaron dos versiones del mismo robot el primero llamado WABOT 1 y tras este crearon WABOT 2.

\vspace{10px}

WABOT 1 fue creado entre los años 1970 y 1973 siendo el primer robot antropomórfico del mundo. El robot tenía control de sus brazos, incorporaba un sistema de visión y análisis del entorno y un sistema conversacional. Éste era capaz de medir su posición con respecto a objetos de la sala y medir las distancias hasta ellos. Además era capaz de andar y coger objetos. En aquel momento, al ser el robot capaz de hablar, se le realizó un test psicológico obteniendo como resultado que WABOT 1 se podría equiparar a un niño de un año de edad.

\vspace{10px}

Tras el éxito del proyecto el grupo de investigación pensó en desarrollar una nueva versión que llamaron WABOT 2 entre los años 1980 y 1984. Esta versión se convirtió en algo mucho menos útil y versátil que WABOT 1, ya que fue diseñado únicamente con el objetivo de que fuera capaz de tocar el piano. Según los propios investigadores esto era interesante pues las artes requieren pensar como un humano y destreza en el movimiento.

\vspace{10px}

Estos dos robots sentaron un precedente abriendo el campo de los robots antropomórficos, los cuales analizaremos en etapas posteriores más cercanas a nuestra época.
