\documentclass[a4paper,10pt]{article}

\usepackage[spanish]{babel}
\usepackage[utf8]{inputenc}
\usepackage[vmargin=2cm,hmargin=2cm]{geometry}
\usepackage{enumerate}
\usepackage{amsmath}
\usepackage{extsizes}
\usepackage{amssymb}
\usepackage{dsfont}
\usepackage{graphicx}
\usepackage{cancel}
\usepackage[usenames]{color}
\usepackage[dvipsnames]{xcolor}
\usepackage{accents}
\usepackage{flushend}

\usepackage[hidelinks]{hyperref}

\setlength\parindent{0pt}

% Carpeta con las imágenes
%\graphicspath{{}}

\begin{document}

\title{\textbf{Historia de la Robótica} \\
		\small{Historia de las Matemáticas}}
\date{}
\author{Ignacio Aguilera Martos, Alicia Rodríguez Gómez, Darío Sierra Martínez\\ \\
	\small Doble Grado en Ingeniería Informática y Matemáticas\\
	\small Facultad de Ciencias, Universidad de Granada
}

\maketitle

\begin{abstract}
	La computación ha tenido un desarrollo muy acelerado desde el siglo XX a nuestros días tomando para ello diferentes objetivos. En primer lugar se intentó automatizar tareas mediante complicadas máquinas que comenzaron siendo completamente mecánicas para proseguir con las actuales electrónicas. En el transcurso de la historia este intento de automatizar las tareas repetitivas y tediosas ha venido acompañado del desarrollo de tecnologías y campos aledaños como la teoría de autómatas o la inteligencia artificial para lograr lo que hoy en día conocemos como robots, o lo que es lo mismo, sistemas artificiales diseñados con un propósito propio. En este trabajo nos proponemos el estudio de la teoría y avances relacionados con la robótica acompañados de un análisis y revisión de la historia de los robots desde los primeros autómatas diseñados en la época griega hasta robots emocionales e inteligentes como Sophia.
\end{abstract}

\newpage

\tableofcontents

\newpage

\section{Teoría de Autómatas, Automatización e Inteligencia Artificial}

\section{Historia Antigua de la Robótica}

\section{Historia Moderna de la Robótica}

%Introducción
\subsection{Introducción}
Acabamos de ver cómo fueron los comienzos de la historia de la robótica desde la época griega hasta los primeros robots industriales de la década de los 60. De aquí en adelante podemos considerar que entramos en la etapa más cercana a nuestros años con un hecho significativo que comenzaremos explicando y marca una diferencia con la etapa previa: la introducción concienzuda de la inteligencia artificial.

Este salto de época viene marcado por la apertura del SRI (Stanford Research Institute) que comienza con la investigación en el campo de la inteligencia artificial tanto en la universidad de Stanford como en la de Edimburgo. Este hecho delimita la historia de la robótica pues es a partir de aquí cuando los investigadores comienzan a dar autonomía e inteligencia a los robots que se pretenden diseñar. Esta herramienta será fundamental en el entendimiento del desarrollo de los robots y las capacidades de los mismos.

%Desarrollo de los laboratorios de IA
\subsection{Inicios de la Inteligencia Artificial}
\subsection{SRI}
SRI o Stanford Research Institute fue una institución amparada bajo la Universidad de Stanford de propósito general en un inicio. Su fundación fue dificultosa y no fue hasta el tercer intento cuando se consiguió en el año 1946.

\vspace{10px}

El SRI hizo investigaciones en variados campos como agricultura, ejército o contaminación pero no fue hasta 1966 cuando el Centro de Inteligencia Artificial comenzó a funcionar bajo el mando de Charles Rosen. El primer proyecto realizado por el SRI en inteligencia artificial fue el Robot Shakey, el primer robot cuya intención no era sólo moverse, si no razonar. Este robot era capaz de analizar el entorno mediante una cámara y procesaba el lenguaje natural para recibir instrucciones. Desarrollando las habilidades de este robot en el desplazamiento por una habitación esquivando los obstáculos que se le imponían en la misma se desarrollaron algoritmos de Visión por Computador tales como filtros y convoluciones para detección de bordes de objetos a ser esquivados y además se desarrollaron algoritmos tan importantes como el A*, el algoritmo por excelencia de cálculo de posibles caminos.

\vspace{10px}

Shakey fue dado por finalizado en el año 1972, siendo ahora los esfuerzos del SRI dirigidos en la creación del Internet para suceder a ARPANET. La primera conexión que se produjo entre dos ordenadores usando el Internet como lo conocemos ahora se produjo en 1977 entre los laboratorios del SRI y la universidad de California UCLA.

\vspace{10px}

Tras estos grandes hitos el SRI se desprendió de la Universidad de Stanford para comenzar su andadura como una empresa, fundándose en ese momento la compañía SRI International, pero la Inteligencia Artificial ya había comenzado.

\subsection{Laboratorios de IA del MIT}
En la misma época el Instituto Tencnológico de Massachusetts o MIT abrió sus laboratorios de inteligencia artificial donde surgieron algunas de las figuras más reconocidas en el ámbito de la computación.

\vspace{10px}

El inicio de la investigación en el ámbito de la computación en el MIT comenzó con figuras tan reseñables como Shannon o Vannevar Bush, el ideólogo del analizador diferencial. En sus comienzos el laboratorio no empezó formándose como tal, si no que se creó un proyecto llamado MAC (Man and Computer) en el que estaban figuras tan importantes como John McCarthy el inventor de Lisp. En un principio el proyecto se dirigió intentando abarcar un amplio rango de temas, como son: la visión por computador, el movimiento mecánico y la manipulación de objetos a partir de máquinas y el manejo del lenguaje natural.

\vspace{10px}
De los despachos en los cuales se llevaban a cabo estas investigaciones surgieron otras figuras como Richard Stallman (creador del proyecto GNU) e invenciones como las máquinas Lisp (máquinas que ejecutaban muy eficientemente Lisp).


%Robotics Institute
\subsection{Robotics Institute}
La Inteligencia Artificial y el desarrollo de la electrónica fue dando paso cada vez con mas fuerza a una nueva rama de la computación: la robótica.

\vspace{10px}

Esta rama es completamente transversal, puesto que aglomera a todos los departamentos relacionados con la computación desde hardware a diseño de software pasando por la visión por computador o la Inteligencia Artificial. Sobre el año 1979 la Universidad Carnegie Mellon en Pittsburgh se dió cuenta del potencial desarrollo de la robótica en un futuro, por lo que decidió abrir un nuevo departamento en la misma que llamó Robotics Institute o Instituto de Robótica, siendo la primera Universidad en tener un departamento dedicado únicamente a la robótica.

\vspace{10px}

Dos robots que podemos destacar de este instituto fueron los vehículos Sandstorm y Highlander dos vehículos completamenta autónomos diseñados para relizar un rallie por un desierto llegando entre puntos del mismo sin ninguna intervención humana. Actualmente los avances logrados en la implementación de estos dos vehículos están siendo implementados en las marcas de coches actuales para intentar lograr cada vez una mayor independización de los conductores de los mismos.

\vspace{10px}

Actualmente el Robotics Institute ya no es sólo un departamento de la Universidad Carnegie Mellon, sino una compañía en si misma con un presupuesto de unos 65 millones de dólares anuales. Actualmente algunos de los campos de trabajo en los que se centran son: robots relacionados con la agricultura, robots industriales, robots médicos, robots asistentes y robots relacionados con el transporte de personas y mercancías.


%WABOT
\subsection{El primer robot antropomórfico: WABOT}
Hasta este momentos los robots habían sido proyectos que implementaban funcionalidades concretas tales como moverse, ver, intentar hablar, reemplazar a los humanos en algunas tareas repetitivas, etc.

\vspace{10px}

En la Universidad de Waseda en Japón, se creó el primer robot con forma antropomórfica llamado WABOT 1. Según los integrantes del proyecto si quisieramos que un robot nos ayudase como un asistente sería conveniente que tuviera forma antropomórfica para que nos resultase más familiar, por lo que se plantearon crear un robot asistente inteligente con forma humana. En este proyecto se desarrollaron dos versiones del mismo robot el primero llamado WABOT 1 y tras este crearon WABOT 2.

\vspace{10px}

WABOT 1 fue creado entre los años 1970 y 1973 siendo el primer robot antropomórfico del mundo. El robot tenía control de sus brazos, incorporaba un sistema de visión y análisis del entorno y un sistema conversacional. Éste era capaz de medir su posición con respecto a objetos de la sala y medir las distancias hasta ellos. Además era capaz de andar y coger objetos. En aquel momento, al ser el robot capaz de hablar, se le realizó un test psicológico obteniendo como resultado que WABOT 1 se podría equiparar a un niño de un año de edad.

\vspace{10px}

Tras el éxito del proyecto el grupo de investigación pensó en desarrollar una nueva versión que llamaron WABOT 2 entre los años 1980 y 1984. Esta versión se convirtió en algo mucho menos útil y versátil que WABOT 1, ya que fue diseñado únicamente con el objetivo de que fuera capaz de tocar el piano. Según los propios investigadores esto era interesante pues las artes requieren pensar como un humano y destreza en el movimiento.

\vspace{10px}

Estos dos robots sentaron un precedente abriendo el campo de los robots antropomórficos, los cuales analizaremos en etapas posteriores más cercanas a nuestra época.


%Brazos robóticos
\subsection{Brazos Robóticos}
\subsubsection{Stanford Arm}

Los brazos robóticos han sido un gran pilar en el desarrollo de la robótica por su gran utilidad. Actualmente rara es la fábrica que no emplea brazos robóticos para hacer la producción mucho más ágil e incluso segura para los operadores de la misma. Este camino fue iniciado en el año 1969 en la Universidad de Stanford por Victor Scheinman.

\vspace{10px}

El primer brazo robótico fue desarrollado bajo el proyecto Hand-Eye en los laboratorios de Inteligencia Artificial de la Universidad de Stanford. El Stanford Arm fue diseñado tras los intentos de hacer brazos operables mediante un humano desempeñados en dicha Universidad. Los intentos precedentes incluían modificaciones de brazos ortopédicos e incluso un modelo hidráulico que resultaba muy peligroso por la rapidez de sus movimientos. El Stanford Arm se creó sin tener una forma antropomórfica y con 6 grados de libertad manipulado de forma completamente eléctrico.

\vspace{10px}

El robot estaba controlado mediante cámaras y elementos parecidos a joysticks gracias a que el brazo llevaba incorporados potenciómetros y sensores que permitían el control del mismo.

\subsubsection{T3}

Tras los grandes avances del Stanford Arm la compañía Cincinnati Milacron Corp. decició desarrolla en 1973 el primer brazo robótico pensado para incorporarlo dentro del mercado industrial. El encargado dentro de la compañía de elaborar el diseño del brazo fue Richard Hohn.

\vspace{10px}

Este brazo robótico, al igual que el Stanford Arm fue desarrollado con 6 grados de libertad permitiendo un amplio abanico de movimientos que el brazo podía hacer con bastante precisión.

\vspace{10px}

Este brazo ya no tenía que ser controlado directamente por un humano, si no que implementaba un módulo de decisiones y sensores capaz de saber si había completado una cierta tarea y tenía que realizar algún movimiento, por ejemplo coger un elemento de una cinta de transporte y moverlo.


%Inicios los movimientos humanos
\subsection{Mejora en la movilidad de los Robots}
\subsubsection{Introducción}

Tras el desarrollo de robots primigenios con funcionalidad concreta pero fija se planteó la posibilidad de integrar movimiento por el terreno en los mismos para poder realizar nuevas tareas u otras de forma más eficiente. En este ámbito se invirtió para lograr avances en esquivar objetos del escenario o en sistemas para andar basados en animales cuadrúpedos o bípedos como nosotros. A continuación analizamos de forma cronológica estos hechos.

\subsubsection{RB5X de RB Robot Corporation}
Un inicio en la movilidad en los robots fue incorporarles ruedas u orugas como herramientas para su desplazamiento por el suelo. Estos tipos de movimientos eran básicos, tales como hacia delante, hacia atrás o hacia los lados. Un avance en el movimiento inteligente lo protagonizó el Stanford's Cart, el cual fue progamado con sensores para desplazarse de una punta a otra de un recinto esquivando los objetos a su paso. Este movimiento inteligente era realizado gracias a un sistema de Visión por Computador que incluía el mismo. Tras esto surgieron nuevos intentos como el RB5X, un robot de pequeñas dimensiones capaz de desplazarse y evitar obstáculos o, en caso de no poder evitarlos, capaz de reconstruir su trayectoria tras impactar con algún objeto.

\vspace{10px}

El robot se ideó de forma multipropósito, puesto que los módulos que integraba eran programables por lo que las empresas tomaron el diseño y lo adecuaron a las tareas que se querían. El robot tuvo variantes tales como asistentes personales a los cuales les podías dar órdenes como recoger el periódico y a través de un brazo robótico que incluía poder interactuar con el medio. Además se le desarrolló una interfaz oral básica para poder darles las órdenes mediante la voz. Además era capaz de aprender de su entorno.

\subsubsection{Phony Pony}

Tras los intentos de hacer robots que incorporasen ruedas u orugas la Universidad de California del Sur realizó el desarrollo del que se conoce como primer robot cuadrúpedo llamado Phony Pony. El diseño de este robot se realizó en 1968 (anterior a RB5X) por los profesores e investigadores Frank y McGhee.

\vspace{10px}

El robot intentaba imitar las articulaciones de los animáles cuadrúpedos que conocemos de forma que lo desarrollaron con dos articulaciones: la cadera y la rodilla. De esta forma el robot podía mover la pierna hacia delante y además flexionarla. La velocidad del movimiento de las patas era extremadamente lenta, ya que al igual que ocurría con el Stanford Arm (contemporáneo a Phony Pony) los elementos eléctricos empleados en el movimiento estaban muy limitados. El Phony Pony era capaz de imitar comportamientos tales como andar, arrastrarse, agacharse o trotar.

\vspace{10px}

El control de este robot se realizaba de forma remota gracias a la tecnología desarrollada ya en esa época. Internamente el robot estaba diseñado mediante patrones de movimiento, de forma que estaba implementado con un autómata finito que tomaba las transiciones del mismo tras recibir la entrada del mando de control.

\subsubsection{WAP1}
En primer paso que se dió entorno a la idea de realizar un robot bípedo tuvo lugar en 1969 en la Universidad de Waseda en Japón. El robot WAP1 (Waseda's antropomorphic pneumatically-activated pedipulators) fue diseñado por el doctor Ichiro Kato.

\vspace{10px}

El doctor Ichiro Kato trabajaba por aquel momento en el terreno médico y empezó a intentar desarrollar músculos artificiales para poder sustituir músculos defectuosos en sus pacientes. El estudio desarrrolló una serie de dispositivos que al hincharse adquirían las formas que ellos necesitaban para simular el comportamiento de un músculo. Estas investigaciones darían sus frutos plasmándose en el primer robot bípedo creado. Los músculos que permitían el movimiento de WAP1 estaban hechos de goma y se inflaban con actuadores neumáticos. Este primer robot no era capaz más que de andar muy lentamente controlando el flujo de aire que se llevaba hacia las piezas de goma. Estas técnicas hacían que WAP llegara a tardar más de 15 segundos en completar un paso.

\vspace{10px}

En los dos siguientes años el doctor Ichiro Kato mejoró el diseño del WAP1 creado el WAP2 y el WAP3. La segunda versión de estos robots tuvo una mejora en la fuerza que los músculos eran capaces de aplicar sobre la estructura del mismo, pudiendo ahora levantar más peso y tener una estructura más robusta. La última versión fue desarrollada en 1971 y en esta el robot ya era capaz de rotar su movimiento, por lo que ya no tenían que ser movimientos rectos y, además, era capaz de superar pequeños obstáculos tales como ligeras inclinaciones o incluso escaleras de pequeño tamaño. Para poder desarrollar todo esto no sólo fueron necesarios los músculos artificiales del doctor Kato, si no además sus estudios en el control y mejora de la postura con lo que fueron capaces de controlar el centro de gravedad del robot, cosa completamente imprescindible para caminar sobre dos piernas.


\normalsize


% Bibliografía.
%-----------------------------------------------------------------
\onecolumn
\bibliography{EtapaModerna}
\bibliographystyle{plain}
\nocite{*}

\end{document}
