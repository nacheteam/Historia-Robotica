\documentclass[a4paper,10pt]{article}

\usepackage[spanish]{babel}
\usepackage[utf8]{inputenc}
\usepackage[vmargin=2cm,hmargin=2cm]{geometry}
\usepackage{enumerate}
\usepackage{amsmath}
\usepackage{extsizes}
\usepackage{amssymb}
\usepackage{dsfont}
\usepackage{graphicx}
\usepackage{cancel}
\usepackage[usenames]{color}
\usepackage[dvipsnames]{xcolor}
\usepackage{accents}
\usepackage{flushend}

\usepackage[hidelinks]{hyperref}

\setlength\parindent{0pt}

% Carpeta con las imágenes
%\graphicspath{{}}

\begin{document}

\title{\textbf{Historia de la Robótica} \\
		\small{Historia de las Matemáticas}}
\date{}
\author{Ignacio Aguilera Martos, Alicia Rodríguez Gómez, Darío Sierra Martínez\\ \\
	\small Doble Grado en Ingeniería Informática y Matemáticas\\
	\small Facultad de Ciencias, Universidad de Granada
}

\maketitle

\begin{abstract}
	La computación ha tenido un desarrollo muy acelerado desde el siglo XX a nuestros días tomando para ello diferentes objetivos. En primer lugar se intentó automatizar tareas mediante complicadas máquinas que comenzaron siendo completamente mecánicas para proseguir con las actuales electrónicas. En el transcurso de la historia este intento de automatizar las tareas repetitivas y tediosas ha venido acompañado del desarrollo de tecnologías y campos aledaños como la teoría de autómatas o la inteligencia artificial para lograr lo que hoy en día conocemos como robots, o lo que es lo mismo, sistemas artificiales diseñados con un propósito propio. En este trabajo nos proponemos el estudio de la teoría y avances relacionados con la robótica acompañados de un análisis y revisión de la historia de los robots desde los primeros autómatas diseñados en la época griega hasta robots emocionales e inteligentes como Sophia.
\end{abstract}

\newpage

\tableofcontents

\newpage

\section{Teoría de Autómatas, Automatización e Inteligencia Artificial}

\section{Historia Antigua de la Robótica}

\section{Historia Moderna de la Robótica}

%Introducción
\subsection{Introducción}
Acabamos de ver cómo fueron los comienzos de la historia de la robótica desde la época griega hasta los primeros robots industriales de la década de los 60. De aquí en adelante podemos considerar que entramos en la etapa más cercana a nuestros años con un hecho significativo que comenzaremos explicando y marca una diferencia con la etapa previa: la introducción concienzuda de la inteligencia artificial.

Este salto de época viene marcado por la apertura del SRI (Stanford Research Institute) que comienza con la investigación en el campo de la inteligencia artificial tanto en la universidad de Stanford como en la de Edimburgo. Este hecho delimita la historia de la robótica pues es a partir de aquí cuando los investigadores comienzan a dar autonomía e inteligencia a los robots que se pretenden diseñar. Esta herramienta será fundamental en el entendimiento del desarrollo de los robots y las capacidades de los mismos.

%Desarrollo de los laboratorios de IA
\subsection{Inicios de la Inteligencia Artificial}
\subsection{SRI}
SRI o Stanford Research Institute fue una institución amparada bajo la Universidad de Stanford de propósito general en un inicio. Su fundación fue dificultosa y no fue hasta el tercer intento cuando se consiguió en el año 1946.

\vspace{10px}

El SRI hizo investigaciones en variados campos como agricultura, ejército o contaminación pero no fue hasta 1966 cuando el Centro de Inteligencia Artificial comenzó a funcionar bajo el mando de Charles Rosen. El primer proyecto realizado por el SRI en inteligencia artificial fue el Robot Shakey, el primer robot cuya intención no era sólo moverse, si no razonar. Este robot era capaz de analizar el entorno mediante una cámara y procesaba el lenguaje natural para recibir instrucciones. Desarrollando las habilidades de este robot en el desplazamiento por una habitación esquivando los obstáculos que se le imponían en la misma se desarrollaron algoritmos de Visión por Computador tales como filtros y convoluciones para detección de bordes de objetos a ser esquivados y además se desarrollaron algoritmos tan importantes como el A*, el algoritmo por excelencia de cálculo de posibles caminos.

\vspace{10px}

Shakey fue dado por finalizado en el año 1972, siendo ahora los esfuerzos del SRI dirigidos en la creación del Internet para suceder a ARPANET. La primera conexión que se produjo entre dos ordenadores usando el Internet como lo conocemos ahora se produjo en 1977 entre los laboratorios del SRI y la universidad de California UCLA.

\vspace{10px}

Tras estos grandes hitos el SRI se desprendió de la Universidad de Stanford para comenzar su andadura como una empresa, fundándose en ese momento la compañía SRI International, pero la Inteligencia Artificial ya había comenzado.

\subsection{Laboratorios de IA del MIT}
En la misma época el Instituto Tencnológico de Massachusetts o MIT abrió sus laboratorios de inteligencia artificial donde surgieron algunas de las figuras más reconocidas en el ámbito de la computación.

\vspace{10px}

El inicio de la investigación en el ámbito de la computación en el MIT comenzó con figuras tan reseñables como Shannon o Vannevar Bush, el ideólogo del analizador diferencial. En sus comienzos el laboratorio no empezó formándose como tal, si no que se creó un proyecto llamado MAC (Man and Computer) en el que estaban figuras tan importantes como John McCarthy el inventor de Lisp. En un principio el proyecto se dirigió intentando abarcar un amplio rango de temas, como son: la visión por computador, el movimiento mecánico y la manipulación de objetos a partir de máquinas y el manejo del lenguaje natural.

\vspace{10px}
De los despachos en los cuales se llevaban a cabo estas investigaciones surgieron otras figuras como Richard Stallman (creador del proyecto GNU) e invenciones como las máquinas Lisp (máquinas que ejecutaban muy eficientemente Lisp).


%Robotics Institute
\subsection{Robotics Institute}
La Inteligencia Artificial y el desarrollo de la electrónica fue dando paso cada vez con mas fuerza a una nueva rama de la computación: la robótica.

\vspace{10px}

Esta rama es completamente transversal, puesto que aglomera a todos los departamentos relacionados con la computación desde hardware a diseño de software pasando por la visión por computador o la Inteligencia Artificial. Sobre el año 1979 la Universidad Carnegie Mellon en Pittsburgh se dió cuenta del potencial desarrollo de la robótica en un futuro, por lo que decidió abrir un nuevo departamento en la misma que llamó Robotics Institute o Instituto de Robótica, siendo la primera Universidad en tener un departamento dedicado únicamente a la robótica.

\vspace{10px}

Dos robots que podemos destacar de este instituto fueron los vehículos Sandstorm y Highlander dos vehículos completamenta autónomos diseñados para relizar un rallie por un desierto llegando entre puntos del mismo sin ninguna intervención humana. Actualmente los avances logrados en la implementación de estos dos vehículos están siendo implementados en las marcas de coches actuales para intentar lograr cada vez una mayor independización de los conductores de los mismos.

\vspace{10px}

Actualmente el Robotics Institute ya no es sólo un departamento de la Universidad Carnegie Mellon, sino una compañía en si misma con un presupuesto de unos 65 millones de dólares anuales. Actualmente algunos de los campos de trabajo en los que se centran son: robots relacionados con la agricultura, robots industriales, robots médicos, robots asistentes y robots relacionados con el transporte de personas y mercancías.



\normalsize


% Bibliografía.
%-----------------------------------------------------------------
\onecolumn
\bibliography{EtapaModerna}
\bibliographystyle{plain}
\nocite{*}

\end{document}
